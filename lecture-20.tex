\documentclass[a4paper,english,12pt]{article}
\usepackage{%
	amsmath,%
	amsfonts,%
	amssymb,%
	amsthm,%
	hyperref,%
	url,%
	latexsym,%
	epsfig,%
	graphicx,%
	psfrag,%
	subfigure,%	
	color,%
	tikz,%
	pgf,%
	pgfplots,%
	pgfplotstable,%
	pgfpages,%
	proofs%
}

\usepgflibrary{shapes}
\usetikzlibrary{%
  arrows,%
	backgrounds,%
	chains,%
	decorations.pathmorphing,% /pgf/decoration/random steps | erste Graphik
	decorations.text,%
	matrix,%
  positioning,% wg. " of "
  fit,%
	patterns,%
  petri,%
	plotmarks,%
  scopes,%
	shadows,%
  shapes.misc,% wg. rounded rectangle
  shapes.arrows,%
	shapes.callouts,%
  shapes%
}

\theoremstyle{plain}
\newtheorem{thm}{Theorem}[section]
\newtheorem{lem}[thm]{Lemma}
\newtheorem{prop}[thm]{Proposition}
\newtheorem{cor}[thm]{Corollary}

\theoremstyle{definition}
\newtheorem{defn}[thm]{Definition}
\newtheorem{conj}[thm]{Conjecture}
\newtheorem{exmp}[thm]{Example}
\newtheorem{assum}[thm]{Assumptions}

%\theoremstyle{remark}
\newtheorem{rem}{Remark}
\newtheorem{note}{Note}

\makeatletter
\def\th@plain{%
  \thm@notefont{}% same as heading font
  \itshape % body font
}
\def\th@definition{%
  \thm@notefont{}% same as heading font
  \normalfont % body font
}
\makeatother
\date{}

%opening
\title{Lecture 20: Compactness}
\author{Parimal Parag}

\begin{document}
\maketitle

\section{Compact spaces}
\begin{defn}
	A collection $\mathcal{A}$ of subsets of a space $X$ is said to \textbf{cover} $X$, or to be a $\textbf{covering}$ of $X$, if the union of the elements of $\mathcal{A}$ is equal to $X$. It is called an \textbf{open covering} of $X$ if its elements are open subsets of $X$.
\end{defn}
	
\begin{defn}	
	A space $X$ is said to be \textbf{compact} if every open covering $\mathcal{A}$ of $X$ contains a finite sub-collection that also covers $X$.
\end{defn}	

\begin{exmp}
	The real line $\mathbb{R}$ is not compact, for the covering of $\mathbb{R}$ by open intervals $A = \{(n,n+2) | n \in \mathbb{Z}\}$
	contains no finite sub-collection that covers $\mathbb{R}$.
\end{exmp}

\begin{exmp}
	The following subspace of $\mathbb{R}$ is compact:$ X= \{0\} \cup \{1/n | n \in \mathbb{Z_+}\} $
	Given an open covering $\mathcal{A}$ of $X$, there is an element $U$ of $\mathcal{A}$ containing $0$. The set $U$ contains all but finitely many of the points $1/n$; choose, for each point of $X$ not in $U$, an element of $\mathcal{A}$ containing it. The collection consisting of these elements of $\mathcal{A}$, along with the element $U$, is a finite sub-collection of $\mathcal{A}$ that covers $X$.
\end{exmp}

\begin{exmp}
	Any space $X$ containing only finitely many points is necessarily compact, because in this case every open covering of $X$ is finite.
\end{exmp}

\begin{exmp}
	The interval $(0,1]$ is not compact; the open covering
	$	\mathcal{A} = \{(1/n,1] | n \in \mathbb{Z_+}\}  $
	contains no finite sub-collection covering $(0, 1]$. The interval $(0, 1)$ is also not compact by the same argument applies. On the other hand, the interval $[0, 1]$ is compact.
\end{exmp}

We shall prove some general theorems that show us how to construct new
compact spaces out of existing ones.

\begin{lem}
	Let $Y$ be a subspace of $X$. Then $Y$ is compact if and only if every covering of $Y$ by sets open in $X$ contains a finite sub-collection covering $Y$.
\end{lem}
\begin{proof}
	Suppose that $Y$ is compact and $\mathcal{A} = \{A_\alpha\}_{\alpha \in J}$ is a covering of $Y$ by sets open in $X$. Then the collection
	$\{A_\alpha \cap Y | \alpha \in J \}$
	is a covering of $Y$ by sets open in $Y$; hence a finite sub-collection
	$ \{A_{\alpha_1} \cap Y, ... , A_{\alpha_n} \cap Y \}$ 
	covers $Y$. Then $\{A_{\alpha_1}, ...,A_{\alpha_n}\}$ is a sub-collection of $\mathcal{A}$ that covers $Y$. 
	
	Conversely, suppose the given condition holds; we wish to prove $Y$ is compact. Let $\mathcal{A}' = \{A'_\alpha \}$ be a covering of $Y$ by sets open in $Y$. For each $\alpha$, choose a set $A_\alpha$ open in $X$ such that $A'_\alpha = A_\alpha \cap Y$. The collection $\mathcal{A} = \{A_\alpha\}$ is a covering of $Y$ by sets open in $X$. By hypothesis, some finite sub-collection $\{A_{\alpha_1},..., A_{\alpha_n}\}$ covers $Y$. Then $\{A'_{\alpha_1},..., A'_{\alpha_n}\}$ is a sub-collection of $\mathcal{A}'$ that covers $Y$.
\end{proof}

\begin{thm}\label{thm26.2}
	Every closed subspace of a compact space is compact.
\end{thm}
\begin{proof}
	Let $Y$ be a closed subspace of the compact space $X$. Given a covering $\mathcal{A}$ of $Y$ by sets open in $X$, let us form an open covering $\mathcal{B}$ of $X$ by adjoining to $\mathcal{A}$ the single open set $X - Y$, that is, $\mathcal{B} = \mathcal{A} \cup \{X - Y\}$.
	Some finite sub-collection of $\mathcal{B}$ covers $X$. If this sub-collection contains the set $X - Y$, discard $X - Y$; otherwise, leave the sub-collection alone. The resulting collection is a finite sub-collection of $\mathcal{A}$ that covers $Y$.
\end{proof}

\begin{thm}\label{thm26.3}
	Every compact subspace of a Hausdorff space is closed.
\end{thm}
\begin{proof}
	Let $Y$ be a compact subspace of the Hausdorff space $X$. We shall prove that $X - Y$ is open, so that Y is closed.\\
	Let $x_0$ be a point of $X - Y$. We show there is a neighborhood of $x_0$ that is disjoint	from $Y$. For each point $y$ of $Y$, let us choose disjoint neighborhoods and of the
	points $x_0$ and $y$, respectively (using the Hausdorff condition). The collection $\{V_y | y \in Y\}$ is a covering of $Y$ by sets open in $X$; therefore, finitely many of them $V_{y_1},...,V_{y_n}$ cover $Y$. The open set $V = V_{y_1} \cup ... \cup V_{y_n}$ contains $Y$, and it is disjoint from the open set $U = U_{y_1} \cap ... \cap U_{y_n}$ formed by taking the intersection of the corresponding neighborhoods of $x_0$. For if $z$ is a point of $V$, then $z \in V_{y_i}$ for some $i$, hence $z \notin U_y$, and so $z \notin U$. Then $U$ is a neighborhood of $x_0$ disjoint from $Y$, as desired. 
\end{proof}

\begin{lem}
	If $Y$ is a compact subspace of the Hausdorff space $X$ and $x_0$ is not in $Y$, then there exist disjoint open sets $U$ and $V$ of $X$ containing $x_0$ and $Y$, respectively.
\end{lem}

\begin{exmp}
	Once we prove that the interval $[a,b]$ in $\mathbb{R}$ is compact, it follows from	Theorem \ref{thm26.2} that any closed subspace of $[a,b]$ is compact. On the other hand, it follows from Theorem \ref{thm26.3} that the intervals $(a, b]$ and $(a, b)$ in $\mathbb{R}$ cannot be compact because they are not closed in the Hausdorff space $\mathbb{R}$.
\end{exmp}

\begin{exmp}
	Consider the finite complement topology on the real line. The only proper subsets of $\mathbb{R}$ that are closed in this topology are the finite sets. But every subset of $\mathbb{R}$ is compact in this topology, as can be checked.
\end{exmp}

\begin{thm}
	The image of a compact space under a continuous map is compact.
\end{thm}
\begin{proof}
	Let $f: X \rightarrow Y$ be continuous; let $X$ be compact. Let $A$ be a covering of the set $f(X)$ by sets open in $Y$. The collection $\{f^{-1} | A \in \mathcal{A}\}$ is a collection of sets covering $X$; these sets are open in $X$ because $f$ is continuous. Hence finitely many of them, say $f^{-1}(A_1),...,f^{-1}(A_n)$, cover $X$. Then, the sets $A_1,...,A_n$ cover $f(X)$.
\end{proof}

\begin{thm}
	Let $I : X \rightarrow Y$ be a bijective continuous function. If $X$ is compact and $Y$ is Hausdorff, then $f$ is a homeomorphism.
\end{thm} 
\begin{proof}
	 We shall prove that images of closed sets of $X$ under $f$ are closed in $Y$; this will prove continuity of the map $f^{-1}$. If $A$ is closed in $X$, then $A$ is compact, by Theorem \ref{thm26.2}. Therefore, by the theorem just proved, $f(A)$ is compact. Since $Y$ is Hausdorff, $f(A)$ is closed in $Y$, by Theorem \ref{thm26.3}.
\end{proof}

\begin{thm}
	The product of finitely many compact spaces is compact.
\end{thm}

\begin{lem}[The tube lemma]
	Consider the product space $X \times Y$, where $Y$ is compact. If $N$ is an open set of $X \times Y$ containing the slice $x_0 \times Y$ of $X \times Y$, then $N$ contains some tube $W \times Y$ about $x_0 \times Y$, where $W$ is a neighborhood of in $X$.
\end{lem}
\begin{proof}
	Let us cover $x_0 \times Y$ by basis elements $U \times V$ lying in $N$. The space $x_0 \times Y$ is compact being homeomorphic to $Y$, we can find finite sub-cover for $x_0 \times Y$ as $U_1 \times V_1,...,U_n \times V_n$. We can assume $U_i \times V_i$ intersects $x_0 \times Y$, $W:=\cap_{(i=1)}^n U_i$ is neighborhood of $x_0$. The chosen $\{U_i \times V_i\}$ cover $W \times Y$ and lie in $N$.
\end{proof}

\begin{thm}
	The product of finitely many compact spaces is compact.
\end{thm}

\begin{proof}
	Let $X$ and $Y$ be compact spaces. Let $\mathcal{A}$ be an open covering of $X \times Y$. Given $x_0 \in X$, $x_0 \times Y$ is compact and may be covered by finitely many elements of $\mathcal{A}$, say, $A_1,..., A_m$. Their union $N=\cup_{i=1}^n A_i$ is an open set containing $x_0 \times Y$. We can find a tube $W \times Y$ about $x_0 \times Y$ where $W$ is open in $X$. Then, $W \times Y$ is covered by finitely many elements $A_1,...,A_m$.
	For each $x \in X$ find neighborhood of $x$ such that tube $W_x \times Y$ is covered by finitely many elements of $\mathcal{A}$. Now $\{W_x : x \in X\}$ is an open covering of X, hence, there exists finite sub-cover $\{W_1,..., W_k\}$ that covers $X$. Union of the tubes $W_1 \times Y,..., W_k \times Y$ covers $X \times Y$.
	Similar argument can be used to complete the proof by induction.
\end{proof}
\end{document}