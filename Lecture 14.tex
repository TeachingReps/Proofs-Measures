\documentclass[a4paper,english,12pt]{article}
\usepackage{%
	amsmath,%
	amsfonts,%
	amssymb,%
	amsthm,%
	hyperref,%
	url,%
	latexsym,%
	epsfig,%
	graphicx,%
	psfrag,%
	subfigure,%	
	color,%
	tikz,%
	pgf,%
	pgfplots,%
	pgfplotstable,%
	pgfpages,%
	proofs%
}

\usepgflibrary{shapes}
\usetikzlibrary{%
  arrows,%
	backgrounds,%
	chains,%
	decorations.pathmorphing,% /pgf/decoration/random steps | erste Graphik
	decorations.text,%
	matrix,%
  positioning,% wg. " of "
  fit,%
	patterns,%
  petri,%
	plotmarks,%
  scopes,%
	shadows,%
  shapes.misc,% wg. rounded rectangle
  shapes.arrows,%
	shapes.callouts,%
  shapes%
}

\theoremstyle{plain}
\newtheorem{thm}{Theorem}[section]
\newtheorem{lem}[thm]{Lemma}
\newtheorem{prop}[thm]{Proposition}
\newtheorem{cor}[thm]{Corollary}

\theoremstyle{definition}
\newtheorem{defn}[thm]{Definition}
\newtheorem{conj}[thm]{Conjecture}
\newtheorem{exmp}[thm]{Example}
\newtheorem{assum}[thm]{Assumptions}

%\theoremstyle{remark}
\newtheorem{rem}{Remark}
\newtheorem{note}{Note}

\makeatletter
\def\th@plain{%
  \thm@notefont{}% same as heading font
  \itshape % body font
}
\def\th@definition{%
  \thm@notefont{}% same as heading font
  \normalfont % body font
}
\makeatother
\date{}

% Title Page
\title{\bf Lecture 14: The Order Topology}
\author{}

\begin{document}

\maketitle
% \begin{abstract}
% \end{abstract}
\section{The Order Topology}
If X is a simply ordered set, there is a standard topology for X, defined using the order relation. It is called "Order Topology".\\
\begin{flushleft}
{\bf Definition}: Suppose X is a set having order relation $<$. Given a$<$b $\epsilon$ X, there are four subsets of X called intervals determined by a and b. They are as follows :\\
$(a,b)=\{x\mid a<x<b\}$\\
$(a,b]=\{x\mid a<x\leq b\}$\\
$[a,b)=\{x\mid a\leq x<b\}$\\
$[a,b]=\{x\mid a\leq x\leq b\}$\\
\vspace{2mm}
{\bf Definition}: Let X be a simple order relation.Assume that X has more than one element.Let $\mathbb{B}$ be the collection of all sets of the following types:\\
\vspace{2mm}
{\bf 1}.All open intervals (a,b) in X.\\
{\bf 2}.All intervals of the form [$a_0$,b), where $a_0$ is the smallest element (if any) of X.\\
{\bf 3}.All intervals of the form (a,$b_0$], where $b_0$ is the largest element (if any) of X.\\
\vspace{2mm}
The collection $\mathbb{B}$ is a basis for a topology on X, which is called the order topology.\\
\vspace{1mm}
{\bf Remark}: If X has no smallest then there are no sets of type {\bf 2} and if X has no largest element then there are no sets of type {\bf 3}.


\end{flushleft}


\end{document} 
