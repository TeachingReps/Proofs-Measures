\documentclass[a4paper,english,12pt]{article}   	% use "amsart" instead of "article" for AMSLaTeX format
\usepackage{geometry}                		% See geometry.pdf to learn the layout options. There are lots.
\geometry{letterpaper}                   		% ... or a4paper or a5paper or ... 
%\geometry{landscape}                		% Activate for rotated page geometry
%\usepackage[parfill]{parskip}    		% Activate to begin paragraphs with an empty line rather than an indent
\usepackage{graphicx}				% Use pdf, png, jpg, or eps§ with pdflatex; use eps in DVI mode
								% TeX will automatically convert eps --> pdf in pdflatex		
\usepackage{amssymb}
\usepackage{caption}
\usepackage{multicol}

\usepackage{%
	amsmath,%
	amsfonts,%
	amssymb,%
	amsthm,%
	hyperref,%
	url,%
	latexsym,%
	epsfig,%
	graphicx,%
	psfrag,%
	subfigure,%	
	color,%
	tikz,%
	pgf,%
	pgfplots,%
	pgfplotstable,%
	pgfpages,%
	proofs%
}

\usepgflibrary{shapes}
\usetikzlibrary{%
  arrows,%
	backgrounds,%
	chains,%
	decorations.pathmorphing,% /pgf/decoration/random steps | erste Graphik
	decorations.text,%
	matrix,%
  positioning,% wg. " of "
  fit,%
	patterns,%
  petri,%
	plotmarks,%
  scopes,%
	shadows,%
  shapes.misc,% wg. rounded rectangle
  shapes.arrows,%
	shapes.callouts,%
  shapes%
}

\theoremstyle{plain}
\newtheorem{thm}{Theorem}[section]
\newtheorem{lem}[thm]{Lemma}
\newtheorem{prop}[thm]{Proposition}
\newtheorem{cor}[thm]{Corollary}

\theoremstyle{definition}
\newtheorem{defn}[thm]{Definition}
\newtheorem{conj}[thm]{Conjecture}
\newtheorem{exmp}[thm]{Example}
\newtheorem{assum}[thm]{Assumptions}

%\theoremstyle{remark}
\newtheorem{rem}{Remark}
\newtheorem{note}{Note}

\makeatletter
\def\th@plain{%
  \thm@notefont{}% same as heading font
  \itshape % body font
}
\def\th@definition{%
  \thm@notefont{}% same as heading font
  \normalfont % body font
}
\makeatother
\date{}
%SetFonts

%SetFonts


\title{Lecture 6: Injectivity, Surjectivity and Bijectivity}
\author{Parimal Parag}
%\date{}							% Activate to display a given date or no date

\begin{document}
\maketitle
%\section{}
%\subsection{}

\section { Injectivity, Surjectivity, Bijectivity}
Let $ f \colon A \to B $ be a function
\begin{defn} [Injectivity] The map $f$ is "injective" (one-to-one/monic/into) if $x \neq y $ implies $f(x) \neq f(y) $  $\forall x,y \in A$.
equivalently if $f(x) = f(y) $ implies $x = y $ $\forall x,y \in A$.
\end{defn}
\begin{defn} [Surjectivity] The map $f$ is "Surjective" (onto/epic) if for every $b \in B$ , there exist some $a \in A $ such that $f(a)=b$. equivalently $f_{*}(A)=B$.
\end{defn}
\begin{defn} [Bijectivity] The map $f$ is "Bijective" if it is both Injective and Surjective.
\end{defn}
\begin{rem}
Injective $\Leftrightarrow$  $\forall b \in B$ $ f^{-1}(\{b\})$ has atmost one element.
\end{rem}
\begin{rem}
Surjective $\Leftrightarrow$  $\forall b \in B$ $ f^{-1}(\{b\})$ has atleast one element.
\end{rem}
\begin{rem}
Bijective $\Leftrightarrow$  $\forall b \in B$ $ f^{-1}(\{b\})$ has exactly one element.
\end{rem}
%\subsection{special case}
\begin{exmp}
let $f \colon \mathbb{R} \to \mathbb{R}$ \\
If \textbf{Injective} Horizontal line intersects at atmost one point.\\
If \textbf{Surjective} Horizontal line intersects at atleast one point.\\
If \textbf{Bijective} Horizontal line intersects at exactly one point.\\
\end{exmp}
\begin{lem}
Let $f \colon A \to B$ , $g \colon B \to C$ be functions.
\begin{enumerate}
\item  $f$,$g$ are injective, then $f \circ g$ Injective.
\item  $f$,$g$ are Surjective, then $f \circ g$ Surjective.
\item  $f$,$g$ are Bijective, then $f \circ g$ Bijective.
\end{enumerate}
\end{lem}
\begin{thm}
Let A and B be two non-empty sets and let $f \colon A \to B$ be a function
\begin{enumerate}
\item $f$ has a right inverse iff $f$ is Surjective.
\item $f$ has a left inverse iff $f$ is Injective.
\item $f$ has a inverse iff $f$ is Bijective.
\end{enumerate}
\end{thm}
\begin{proof} Suppose $f$ has a right inverse $g$, then 
$f \circ g$ = $I_{B}$. We will show $f$ is Surjective.
Let $b \in B$ We need to find an element $a \in A $ such that $f(a)=b$. Let $a=g(b)$ then $f(g(b))$ = $(f \circ g)(b)$ =$I_{B}(b)=b$.
\par
Suppose $f$ is Surjective. We wish to show that a map $g \colon B \to A$ such that $f$ has a right inverse , i.e there exists a map
$g \colon B \to A$ such that $f \circ g$=$I_{B}$.\\
We define $g$ as follows. For each $b \in B$ , There is at least one $a \in A$ such that $f(a)=b$. Let $g(b)=a$ for some $a \in f^{-1}(\{b\})$.
Now $(f \circ g)(b)=b$ for all $b \in B$ by definition. Hence $f \circ g = I_{B}$.

\end{proof} 
\begin{thm} Let A and B be non empty sets and let $f \colon A \to B$ be a function.
\begin{enumerate}
\item the functiomn $f$ is Surjective iff $g \circ f = h \circ f$ implies $g=h$ for all functions $g,h \colon B \to X$ for all sets X.
\item The function $f$ is Injective iff $f \circ g = f \circ h$ implies $g=h$ for all functions $g,h \colon Y \to A$ for all sets Y.
\end{enumerate}
\begin{proof}
Assume $f$ Surjective Let $g,h \colon B \to X$ such that $g \circ f = h \circ f$ for some set X. By Theorem 1.5.1 , function $f$ has right inverse $q \colon B \to A$. By Assosiativity We have $(g \circ f) \circ q = g \circ (f \circ q) $ and thus $g \circ I_{B} = h \circ I_{B}$ and so $g=h$.\\
We assume $f$ not Surjective. Let $b \in B$ such that $f^{-1}(\{b\})= \phi$. Let $X={1,2}$ define $g,h \colon B \to X$ by $g(y)=1$ $\forall y \in B$ and $h(y)=1$ $ \forall y \in B \backslash \{b\}$ and $h(b)=2$. It can be verified that $g \circ f = h \circ f$, even though $g \neq h$.
\end{proof}
\end{thm}
\subsection{Sets of functions}
\begin{defn} Let A and B be sets. The set $F(A,B)$ is defined to be the set of all functions $f \colon A \to B$.
\[ F(A,B) = \{f \colon A \to B |\, f \,is function \} \]
\end{defn}
\begin{exmp} A=\{1,2\} , B=\{x,y\} \\
$F(A,B) = \{ (f,g,h,k)\}\\$
$f=\{(1,x),(2,x)\}$ , $g=\{(1,y),(2,y)\}$ , $h=\{(1,x),(2,y)\}$ , $k=\{(1,y),(2,x)\}$
\end{exmp}
\begin{figure}[]
\centering
\scalebox{.8}{
 
 \begin{tikzpicture}[ele/.style={fill=black,circle,minimum width=.8pt,inner sep=1pt},every fit/.style={ellipse,draw,inner sep=-1pt}]
  \node[ele,label=left:$1$] (a1) at (0,4) {};    
  \node[ele,label=left:$2$] (a2) at (0,3) {};    
 

  \node[ele,,label=right:$x$] (b1) at (4,4) {};
  \node[ele,,label=right:$y$] (b2) at (4,3) {};
 

  \node[draw,fit= (a1) (a2) ,minimum width=2cm] {} ;
  \node[draw,fit= (b1) (b2) ,minimum width=2cm] {} ;  
  \draw[->,thick,shorten <=2pt,shorten >=2pt] (a1) -- (b1);
  \draw[->,thick,shorten <=2pt,shorten >=2pt] (a2) -- (b1);
 
 \end{tikzpicture} \begin{tikzpicture}[ele/.style={fill=black,circle,minimum width=.8pt,inner sep=1pt},every fit/.style={ellipse,draw,inner sep=-1pt}]
  \node[ele,label=left:$1$] (a1) at (0,4) {};    
  \node[ele,label=left:$2$] (a2) at (0,3) {};    
 

  \node[ele,,label=right:$x$] (b1) at (4,4) {};
  \node[ele,,label=right:$y$] (b2) at (4,3) {};
 

  \node[draw,fit= (a1) (a2) ,minimum width=2cm] {} ;
  \node[draw,fit= (b1) (b2) ,minimum width=2cm] {} ;  
  \draw[->,thick,shorten <=2pt,shorten >=2pt] (a1) -- (b2);
  \draw[->,thick,shorten <=2pt,shorten >=2pt] (a2) -- (b2);
 
 \end{tikzpicture}}
\caption{functions f,g}
\label{}
\end{figure}
\begin{figure}[]
\centering
\scalebox{.8}{\begin{tikzpicture}[ele/.style={fill=black,circle,minimum width=.8pt,inner sep=1pt},every fit/.style={ellipse,draw,inner sep=-1pt}]
  \node[ele,label=left:$1$] (a1) at (0,4) {};    
  \node[ele,label=left:$2$] (a2) at (0,3) {};    
 

  \node[ele,,label=right:$x$] (b1) at (4,4) {};
  \node[ele,,label=right:$y$] (b2) at (4,3) {};
 

  \node[draw,fit= (a1) (a2) ,minimum width=2cm] {} ;
  \node[draw,fit= (b1) (b2) ,minimum width=2cm] {} ;  
  \draw[->,thick,shorten <=2pt,shorten >=2pt] (a1) -- (b1);
  \draw[->,thick,shorten <=2pt,shorten >=2pt] (a2) -- (b2);
 
 \end{tikzpicture} \begin{tikzpicture}[ele/.style={fill=black,circle,minimum width=.8pt,inner sep=1pt},every fit/.style={ellipse,draw,inner sep=-1pt}]
  \node[ele,label=left:$1$] (a1) at (0,4) {};    
  \node[ele,label=left:$2$] (a2) at (0,3) {};    
 

  \node[ele,,label=right:$x$] (b1) at (4,4) {};
  \node[ele,,label=right:$y$] (b2) at (4,3) {};
 

  \node[draw,fit= (a1) (a2) ,minimum width=2cm] {} ;
  \node[draw,fit= (b1) (b2) ,minimum width=2cm] {} ;  
  \draw[->,thick,shorten <=2pt,shorten >=2pt] (a1) -- (b2);
  \draw[->,thick,shorten <=2pt,shorten >=2pt] (a2) -- (b1);
 
 \end{tikzpicture}}
\caption{functions h,k}
\label{}
\end{figure}

\begin{exmp} $F(\mathbb{R},\mathbb{R}) \supset C(\mathbb{R},\mathbb{R}) \supset D(\mathbb{R},\mathbb{R})$\\
$K \colon D(\mathbb{R},\mathbb{R}) \to F(\mathbb{R},\mathbb{R})$\\
$K(f)=f'$
\end{exmp}
\begin{exmp}
$f \in F(\mathbb{N},\mathbb{R})$ \\
$f(1),f(2),.....$ is a sequences of real numbers.
\end{exmp}
\begin{lem}
Let A,B,C,D be sets suppose that there are bijective maps $f \colon A \to C$ and $g \colon B \to D$ then there is a bijecition map between
$F(A,B)$ and $F(C,D)$.
\end{lem}
\begin{proof} Since $f$ and $g$ are bijective maps , They have inverse maps $f^{-1}$ and $g^{-1}$ respectively.
\begin{figure}[hhhh]
\centering
\scalebox{.8}{\begin{tikzpicture}
  \matrix (m) [matrix of math nodes,row sep=3em,column sep=4em,minimum width=2em]
  {
     A & B \\
     C & D \\};
  \path[-stealth]
    (m-1-1) edge node [left] {f} (m-2-1)
            edge node [below] {h} (m-1-2)
    (m-2-1) edge node [below] {$\Phi (h) $} (m-2-2)
     (m-1-2) edge node [right] {g} (m-2-2);
 \end{tikzpicture}}
\caption{}
\label{}
\end{figure}\\
Define $\Phi \colon F(A,B) \to F(C,D)$ by $\Phi (h)=g \circ h \circ f^{-1}$. for all $h \in F(A,B)$ \\
It's easy to see that $\Phi (h) \in F(C,D)$ for all $h \in F(A,B)$. We need to show $\Phi$ is bijective.\\
\textbf{Injective:} Let $h,k \in F(A,B)$ and suppose $\Phi (h)=\Phi (k)$ Then $g \circ h \circ f^{-1} = g \circ k \circ f^{-1}$, hence $h=k$ , $(g^{-1} \circ g \circ h \circ f^{-1} \circ f)=(g^{-1} \circ g \circ h \circ f^{-1} \circ f)$\\
\textbf{Surjective:} Let $r \in F(C,D)$ and define $t=g^{-1} \circ r \circ f$ clearly $t \in F(A,B)$ , $\Phi (t)=g \circ (g^{-1} \circ r \circ f) \circ f^{-1}=r$
\end{proof}
\begin{prop} Let A be a non empty set. Then there is a bijective map from $F(A,\{0,1\})$ to $P(A)$.
\begin{proof} Let $\Phi \colon P(A) \to F(A,\{0,1\})$ be defined as\\
\[ [\Phi (s)](x)\, \, = 1  \quad  x \in S \] 
$\qquad \qquad \qquad \qquad \qquad \qquad \qquad \qquad \qquad=0 \quad  x \in A \backslash S$\\
for all $S \subseteq A , \Phi (S) \colon A \to \{0,1\}$\\
\textbf{Injectivity:} Let $S,T \in P(A)$ and suppose $\Phi (S)=\Phi (T)$ ,We will show $S=T$\\
$y \in S , [ \Phi (S)](y)=1=[ \Phi (T)](y)$ then $y \in T$. Hence $S \subseteq T$. Similarly $T \subseteq S$.\\
\textbf{Surjectivity:} Let $f \in F(A,\{0,1\})$, Define $s \in P(A)$ such that $S=\{x \in A | f(x)=1\}=f^{-1}(\{1\})$. We will show that 
$\Phi (S)=f$.\\ For all $x \in S$ , $\Phi (S)(x)=1=f(x)$ , $x \notin S , \Phi(S)(x)=0=f(x)$ .
\end{proof}
\end{prop}
\end{document}  