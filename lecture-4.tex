\documentclass[a4paper,english,12pt]{article}
\usepackage{%
	amsmath,%
	amsfonts,%
	amssymb,%
	amsthm,%
	hyperref,%
	url,%
	latexsym,%
	epsfig,%
	graphicx,%
	psfrag,%
	subfigure,%	
	color,%
	tikz,%
	pgf,%
	pgfplots,%
	pgfplotstable,%
	pgfpages,%
	proofs%
}

\usepgflibrary{shapes}
\usetikzlibrary{%
  arrows,%
	backgrounds,%
	chains,%
	decorations.pathmorphing,% /pgf/decoration/random steps | erste Graphik
	decorations.text,%
	matrix,%
  positioning,% wg. " of "
  fit,%
	patterns,%
  petri,%
	plotmarks,%
  scopes,%
	shadows,%
  shapes.misc,% wg. rounded rectangle
  shapes.arrows,%
	shapes.callouts,%
  shapes%
}

\theoremstyle{plain}
\newtheorem{thm}{Theorem}[section]
\newtheorem{lem}[thm]{Lemma}
\newtheorem{prop}[thm]{Proposition}
\newtheorem{cor}[thm]{Corollary}

\theoremstyle{definition}
\newtheorem{defn}[thm]{Definition}
\newtheorem{conj}[thm]{Conjecture}
\newtheorem{exmp}[thm]{Example}
\newtheorem{assum}[thm]{Assumptions}

%\theoremstyle{remark}
\newtheorem{rem}{Remark}
\newtheorem{note}{Note}

\makeatletter
\def\th@plain{%
  \thm@notefont{}% same as heading font
  \itshape % body font
}
\def\th@definition{%
  \thm@notefont{}% same as heading font
  \normalfont % body font
}
\makeatother
\date{}
%\usepackage[T1]{fontenc}
%\PassOptionsToPackage{normalem}{ulem}
%\usepackage{ulem}
%\makeatletter
%%%%%%%%%%%%%%%%%%%%%%%%%%%%%% LyX specific LaTeX commands.
%\pdfpageheight\paperheight
%\pdfpagewidth\paperwidth

%\makeatother

\begin{document}

\title{Lecture 4: Set Operations}
\author{}
\maketitle

\section{Set Operations}
New sets can be constructed from existing sets by combining one or multiple of them using set operations. This is somewhat analogous to the the construction of natural and rational numbers. The two most basic set operations that we will now describe correspond to the familiar logical operations ``or'' and ``and''. \\

\begin{defn}[Union] Let $A$ and $B$ be sets. The \textbf{union} of sets $A$ and $B$, denoted $A \cup B$, is the set defined by 
\begin{equation*}
 A \cup B = \{ x | x \in A \text{ or } x \in B \}.
\end{equation*}
\end{defn}
\begin{defn}[Intersection] Let $A$ and $B$ be sets. The \textbf{intersection} of sets $A$ and $B$, denoted $A \cap B$, is the set defined by 
\begin{equation*}
 A \cup B = \{ x | x \in A \text{ and } x \in B \}.
\end{equation*}
\end{defn}

Do the familiar properties of addition and multiplication of numbers (such as commutativity and associativity) also hold for union and intersection of sets? The following theorem shows that such properties do hold, although they are not exactly the same as for addition and multiplication.
\begin{thm}
 Let $A$, $B$ and $C$ be sets. 
\begin{enumerate} [i)]
\item $A \cap B \subseteq A$ and $A \cap B \subseteq B$. If $X$ is a set such that $X \subseteq A$ and $X \subseteq B$, then 
$X \subseteq A \cap B $.
\item $A \subseteq A \cup B $ and $B \subseteq A \cup B $. If Y is a set such that $A \subseteq Y$ and $B \subseteq Y$, then $A \cup B \subseteq Y$.
\item (Commutative Laws) $A \cup B = B \cup A$ and $A \cap B = B \cap A$.
\item (Associative Laws) $(A \cup B) \cup C = A \cup (B \cup C)$ and $(A \cup B) \cup C = A \cap (B \cap C)$. 
\item (Distributive Laws) $A \cap (B \cup C)$ = $(A \cap B) \cup (A \cap C)$ and $A \cup (B \cap C)$ = $(A \cup B) \cap (A \cup C)$
\item (Identity Laws) $A \cup \phi = \phi$ and $A \cap \phi = \phi$.
\item (Idempotent Laws) $A \cup A = A$ and $A \cap A = A$.
\item (Absorption Laws) $A \cup (A \cap B) = A$ and $A \cap (A \cup B) = A$.
\item If $A \subseteq B$, then $A \cup C \subseteq B \cup C$ and $A \cap C \subseteq B \cap C$.
\end{enumerate}
\end{thm}

It is interesting to note that both union and intersection distribute over each other, which is quite different from addition and multiplication of numbers, where multiplication distributes over addition, but not vice versa.

\begin{defn}[Disjoint sets] 
Let $A$ and $B$ be sets. The sets $A$ and $B$ are \textbf{disjoint} if $A \cap B = \phi$.
\end{defn}

\begin{exmp}
Let $E$ be the set of even integers, let $O$ be the set of odd integers and let $P$ be the set of prime numbers. Then $E$ and $O$ are disjoint, 
whereas $E$ and $P$ are not disjoint (because $E \cap P = \{2\}$). Another interesting example of disjoint sets which we will often encounter 
is the partition of a set.
\end{exmp}

\begin{defn}[Set difference] 
Let $A$ and $B$ be sets. The \textbf{set difference} of $A$ and $B$ is denoted by $A \setminus B$ and is defined by 
 \begin{equation}
  A \backslash B = \{ x \; | \;  x \in A \text{ and } x \notin B  \}.
 \end{equation}
The set difference $A \setminus B$ can also be written as $A \cap B^{C}$.
\end{defn}

\begin{exmp}
Let $A = \{a, b, c\}$ and $B = \{b \}$, then $A \setminus B = \{a, c\}$. 
\end{exmp}

The following theorem gives some standard properties of set difference.
\begin{thm}
 Let $A$, $B$ and $C$ be sets.
 \begin{enumerate}[i)]
  \item $A \setminus B \subseteq A$.
  \item $(A \setminus B) \cap B = \phi$.
  \item $A \setminus B = \phi$ iff $A \subseteq B$.
  \item $B \setminus (B \setminus A) = A$ iff $A \subseteq B$.
  \item If $A \subseteq B$, then $A \setminus C \subseteq B \setminus C$.
  \item If $A \subseteq B$, then $(C \setminus A) \supseteq (C \setminus B)$.
  \item (De Morgan's Laws) $C \setminus (A \cup B) = (C \setminus A) \cap (C \setminus B)$ and $C \setminus (A \cap B) = (C \setminus A) \cup (C \setminus B)$.
 \end{enumerate}
\end{thm}

\begin{proof}
\begin{enumerate} [i)]
 \item Follows from definition of set difference.
 \item We can rewrite left hand side as $(A \cap B^{C}) \cap B$ which is equal to 
 $A \cap (B^{C} \cap B)$ using associativity property of $\cap$. Since $B$ and $B^{C}$ are disjoint sets, $A \cap (B^{C} \cap B) = A \cap \phi = \phi$. 
 \item Let us assume that $A \subseteq B$. Let $x$ be an element of $A \backslash B$, which implies $x \in A$ and $x \notin B$. Since $A \subseteq B$, $x \in B$ as well, which contradicts the previous  statement. Hence, there exists no $x$ such that $x \in A \backslash B$, implying that $A \backslash B = \phi$.  For the converse, let us assume that $A \backslash B = \phi$ and let $x \in A$, then, we need to show that $x$ also belongs to the set $B$. Let us suppose that $x \notin B$, then $x \in B^{C}$. Then, it follows that $x \in A \ B$, which is a contradicts the initial assumption that 
 $A \backslash B$ is an empty set. Hence, $x \in B$. Since choice of $x$ was arbitrary, we conclude that $A \subseteq B$.    
 \item Let us assume that $A \subseteq B$. Let $x$ be an element of $B \backslash (B \backslash A)$, implying that $x \in B$ and $x \notin (B \backslash A)$ or equivalently $x \in (B \backslash A)^{C}$
 which translates to $x \in (B^{C} \cap A)$ by using De Morgan's law. Since $x \notin B^{C}$, $x$ must belong to the set $A$. Thus, we have shown that $B \backslash (B \backslash A) \subseteq A$.
To show that $A \subseteq B \backslash (B \backslash A)$, let $y$ be an element of set $A$ and consequently $y$ is also an element of set $B$, since it is given that $A \subseteq B$. Since $y$ is an element of both $A$ and $B$, it does not belong to the set $B \backslash A$. Thus, from definition of set difference, $y$ is an element of $B \backslash (B \backslash A)$.

To prove the converse, let $z$ be an element of the set $A$. Since $B \backslash (B \backslash A) = A$, $z$ is also an element of the set $B \backslash (B \backslash A)$. Then, it follows that $z \in B$, which completes the proof as the choice of $z$ was arbitrary.
 
\item Let $x$ be any arbitrary element of the set $A \backslash C$. This implies that $x \in A$ and $x \notin C$. From $A \subseteq B$, $x $ is also an element of set $B$. Since $x$ belongs to the set $B$ and does not belong to the set $C$, it follows that $x \in B \backslash C$. 

\item Let $x$ be any arbitrary element of the set $C \backslash B$. This implies that $x \in C$ and 
$x \notin B$. We need to show that $x \notin A$. Let us suppose otherwise. Then, from $A \subseteq B$, we have $x \in B$, which contradicts the $x \in C \backslash B$. Hence, $x \notin A$ and therefore $x \in C \backslash A$.     

\item We will show that $C \backslash (A \cap B) = (C \backslash A) \cup (C \backslash B)$, the other equation can be proved similarly, and we omit the details. Let $x \in C \backslash(A \cup B)$. 
Then $x \in C$ and $x \notin A \cup B$. It follows that $x \notin A$ and $x \notin  B$, because 
$x \in A$ or $x \in B$ would imply that $x \in A \cup B$. Because $x \in C$ and $x \notin A$, then $x \in C \backslash A$. Because $x \in C$ and $x \notin B$, then $x \in C \backslash B$. Hence $x \in (C \backslash A)\cap(C \backslash B)$. Therefore $C \backslash (A \cup B) \subseteq  (C \backslash A)\cap (C \backslash B)$.

Now let $y \in (C \backslash A) \cap (C \backslash B)$. Hence $y \in C \backslash A$ and $y \in C \backslash B$. Because $y \in C \backslash A$, it follows that $y \in C$ and $y \notin A$. Because $y \in C \backslash B$, it follows that $y \in C$ and $y \notin B$.
Because $y \notin A$ and $y \notin B$, it follows that $y \notin A \cup B$. Therefore $y  \in C \backslash (A \cup B)$. Hence $(C \backslash A) \cap (C \backslash B)  \subseteq C \backslash (A \cup B)$.

We conclude that $C \backslash (A \cup B) = (C \backslash A) \cap (C \backslash B)$. 
\end{enumerate}
\end{proof}


There is one more fundamental way of forming new sets out of old by taking their product. Think of how the plane $\Real^{2}$ is coordinatized by ordered pairs of real
numbers. In the following definition we make use of the notion of an ordered pair of elements, denoted $\left(a, b \right)$, where $a$ and $b$ are elements of
some given sets. Unlike a set $\{a, b\}$, where the order of the elements does not matter (so that $\{a, b\} = \{b, a\}$), in an ordered pair the order of the 
elements does matter. Note that the ordered pair $\left(a, b \right)$ equals the ordered pair $\left(c, d \right)$ if and only if $a = c$ and $b = d$. 

\begin{defn}[Cartesian product]
Let $A$ and $B$ be sets. The Cartesian \textbf{product} of $A$ and $B$, denoted 
$A \times B = \{ (a, b) \; | \; a \in A \text{ and } b \in B \}$, 
where $(a, b)$ denotes an ordered pair.
\end{defn}

\begin{exmp} Following are simple examples of Cartesian products of real number line.
\begin{enumerate}[(i)]
\item $\Real^{2} = \Real \times \Real$.
\item $\Real^{n} = \Real \times \Real \dots \times \Real$.
\end{enumerate}
\end{exmp}

The following theorem gives some properties of the Cartesian product.
\begin{thm}
 Let $A, B, C, D$ be sets.
 \begin{enumerate}[i)]
  \item If $A \subseteq B$ and $C \subseteq D$, then $A \times C \subseteq B \times D$
  \item (Distributive Laws)
  \begin{enumerate}
   \item $A \times (B \cup C) = (A \times B) \cup (A \times C)$ 
   \item $(B \cup C) \times A = (B \times A) \cup (C \times A)$
   \item $A \times (B \cap C) = (A \times B) \cap (A \times C)$
   \item $(B \cap C) \times A = (B \times A) \cap (C \times A)$
  \end{enumerate}
  \item $A \times \phi = \phi$ and $\phi \times A = \phi$.
  \item $(A \cap B) \times (C \cap D) = (A \times C) \cap (B \times D)$.
\end{enumerate}

\end{thm}

\section{Indexed Families of Sets}
Let $\{A_i\}$ be a set for each $i$ in $\{1,2,\ldots,n\}$. We set the following notation for union and intersection of these sets as, 
\begin{align*}
 \bigcup_{i = 1}^{n} A_{i} = A_{1} \cup A_{2} \cup \cdots \cup A_{n} \text{ and }& 
 \bigcap_{i = 1}^{n} A_{i} = A_{1} \cap A_{2} \cup \cdots \cap A_{n} \text{ respectively}.
\end{align*}

\begin{defn}[Indexed Family of Sets]
 Let $I$ be a non-empty set. Suppose there is a set $A_{i} \subseteq U$ for each element $i \in I$. Such a collection is called a ``family of sets indexed by $I$''. The set $I$ is called the \textbf{indexing set}, for this family of sets. 
\end{defn}

We denote the family of sets by $\{A_{i} \;:\; i \in I\}$. Union and intersection for this family of sets are denoted by $\bigcup_{i \in I}A_{i}$ and $\bigcap_{i \in I}A_i$ respectively, and defined as
\begin{align*}
 \bigcup_{i = \in I}A_{i} &= \{x \; | \; x \in A_{i} \text{ for some } i \in I \}, \text{ and }\\
 \bigcap_{i = \in I}A_{i} &= \{x \; | \; x \in A_{i} \text{ for all } i \in I \} \text{ respectively}.
 \nonumber
\end{align*}

\begin{thm}
Let $I$ be a non-empty set, $\{ A_{i} : i \in I \}$ be a family of sets indexed by $I$, and $B$ be a set.
\begin{enumerate}[i)]
 \item $\displaystyle \bigcap_{i \in I} A_{i}  \subseteq A_{k}$ for all $k \in I$. If $B \subseteq A_{k}$ for all $k \in I$, then $B\displaystyle \subseteq \bigcap_{i \in I}A_{i}$.
 \item $A_{k} \subseteq \displaystyle \bigcup_{i \in I} A_{i}$ for all $k \in I$. If $A_{k} \subseteq B$ for all $k \in I$, then $\displaystyle \bigcup_{i \in I}A_{i} \subseteq B$. \\
 \item (Distributive laws) 
		\begin{enumerate}
			\item $\displaystyle B \cap \left( \bigcup_{i \in I}A_{i} \right) = \bigcup_{i \in I} \left( B \cap A_{i} \right)$
			\item $\displaystyle B \cup \left( \bigcap_{i \in I}A_{i} \right) = \bigcap_{i \in I} \left( B \cup A_{i} \right)$
		\end{enumerate}
 \item (De Morgan's laws) 
		\begin{enumerate}
			\item $B \backslash \left( \bigcup_{i \in I}A_{i} \right) = \bigcap_{i \in I} \left( B \backslash A_{i} \right)$
			\item $B \backslash \left(\bigcap_{i \in I}A_{i} \right) = \bigcup_{i \in I} \left( B \backslash A_{i} \right)$
 \end{enumerate}
\end{enumerate}
\end{thm}
\end{document}
