%% LyX 2.1.3 created this file.  For more info, see http://www.lyx.org/.
\documentclass[a4paper,english]{article}
\usepackage[T1]{fontenc}
\PassOptionsToPackage{normalem}{ulem}
\usepackage{ulem}

\makeatletter

%%%%%%%%%%%%%%%%%%%%%%%%%%%%%% LyX specific LaTeX commands.
\pdfpageheight\paperheight
\pdfpagewidth\paperwidth


\makeatother

\usepackage{babel}
\begin{document}

\title{Class 3 11 August 2015}

\maketitle

\section{Rules of inference using quantifiers}

1. \uline{($\forall x$ in $U$) P($x$)} $\hphantom{ttttt}$Universal
instantiation

$\hphantom{ttttt}$P($a$) $\hphantom{tttttttttt}$where $a$ is any
member in $U$\\
2.\uline{($\exists x$ in $U$) P($x$)} $\hphantom{ttttt}$Existential
instantiation

$\hphantom{ttttt}$P($a$) $\hphantom{tttttttttt}$where $a$ is some
member in $U$\\
3.\uline{$\hphantom{ttttt}$P($c$)$\hphantom{ttttttttt}$}$\hphantom{ttttt}$Universal
instantiation

($\forall x$ in $U$) P($x$) $\hphantom{tttt}$where $c$ is any
arbitrary member in $U$\\
4.\uline{$\hphantom{ttttt}$P($d$)$\hphantom{ttttttttt}$}$\hphantom{ttttt}$Existential
instantiation

($\forall x$ in $U$) P($x$) $\hphantom{tttt}$where $d$ is some
member in $U$\\


Eg:-Let\\
$N(x)=$'cat $x$ is nice'\\
$S(x)=$'cat $x$ is smart'\\
$C(x)=$'cat $x$ likes chopped liver'\\
$T(x)=$'cat $x$ is siamese'\\
Let the arguments be 

1.Every cat that is nice and smart likes chopped liver i.e $\forall x$
in $U$ $[N(x)\wedge S(x)\rightarrow C(x)]$

2.Every siamese cat is nice i.e $\forall x$ in $U$ $[T(x)\rightarrow N(x)]$

3.Some siamese cat don't like chopped liver i.e $\exists x$ in $U$
$[T(x)\wedge\lnot C(x)]$\\
And the conclusion is 

There exists a stupid cat i.e $\exists x$ in $U$ $[\lnot S(x)]$\\
We prove the conclusion on the basis of the premises as follows:

4.$T(a)\wedge\lnot C(a)$ by 3,Existential Instantiation

5.$\lnot C(a)$ by simplification of 4

6.$T(a)$ by simplification of 4

7.$T(a)\rightarrow N(a)$ Universal Instantiation by 2

8.$N(a)$ Modus Ponens by 6 and 7

9.$\lnot\lnot N(a)$ Double Negation of 8

10.$(N(a)\wedge S(a)\rightarrow C(a)$ Universal Instantiation from
1

11.$\lnot(N(a)\wedge S(a))$ by Modus Tollen using 10 and 5 

12.$\lnot N(a)\vee\lnot S(a)$ Simplified 11 using De Morgan's law

13.$\lnot S(a)$ ,Considering 9 and 12 together and applying Modus
Tollendo Ponens

14.$\exists x$ in $U$ $[\lnot S(x)]$ Existential Generalization
of 13 and hence proved that a stupid cat exists based on the premises


\section{Strategies for Proofs}

Thales of Miletus (6 BC) is credited with introducing the concepts
of logical proof for abstract propositions.\emph{The elements} , book
by Euclid ,written over 2000 years ago , discusses the various proofs
of geometry based on axioms .So Why do we need proofs? . Human intuition/common
sense is fallible (Sometimes we see what we want to see). Proofs are
are necessary to infallibly ascertain facts ,to explain why things
are true and to get a thorough understanding. So,
\begin{description}
\item [{Mathematical~Proof}] A convincing argument that starts from the
premises and logically deduces the desired conclusion .
\item [{Axioms}] Facts about mathematical objects assumed without proof.
\item [{Axiomatic~system}] Body of knowledge that can be derived from
a set of axioms .
\item [{Eg:}] P.T ``Sum of even numbers are even''
\item [{Ans}] precise statement for the above problem - `` If $n$ and
$m$ are even integer numbers then $n+m$ is an even number .''
\end{description}
Let $n$ and m be even integers , then there exist integers $k$ and
$l$ s.t $n=2k$ and $m=2l$.So $n+m=2(k+l)$ where $(k+l)$ is also
an integer by closure property of integer addition.So $n+m$ is also
an integer multiple of $2$ and hence even and therefore proved .
\begin{description}
\item [{Eg:}] P.T ``$\sqrt{2}$ is an irrational number .''
\item [{Ans}]~
\item [{Defn:}] Let $x$ be a real number.We say $x$ is a rational number
if there exists integers $m$ and $n$ s.t $m\ne0$ and $x=\frac{n}{m}$
and there are no 2 common factors between $n$ and $m$ other than
$1$ or $-1$. 
\item [{Proof}] Let $x$ be a rational number s.t $x=\sqrt{2}$.Then $x^{2}=2$.So
by definition above $\frac{n^{2}}{m^{2}}=2$ or $n^{2}=2m^{2}$.So
$n^{2}$ is an even number , therefore $n$ must be even . Hence there
exists an integer $k$s.t $n=2k$. Thus we have $4k^{2}=2m^{2}$ i.e
$2k^{2}=m^{2}$.Then $m$ must also be even i.e $m$ and $n$ have
$2$ as common factor which contradicts the fact that both must have
no common factor other than $1$ or $-1$ by definition above .Hence
$x=\sqrt{2}$ is an irrational number .
\end{description}

\subsection{Features of Proofs}
\begin{itemize}
\item Should rely completely on the definiton of odd and even integers .
\item Should be written in grammatically correct English .
\item Implicit use of rules of inference
\end{itemize}

\subsection{Types of Proofs }
\begin{description}
\item [{Direct~Proofs~$P\rightarrow Q$}] Assume that $P$ is true and
produce a series of steps eventually leading to $Q$.
\item [{Proof~by~contrapositive~$\lnot Q\rightarrow\lnot P$}] Assume
$Q$ is false and produce a series of steps eventually leading to
$\lnot P$.
\item [{Proof~by~contradiction~$\lnot(P\rightarrow Q)\leftrightarrow P\wedge\lnot Q$}] Assume
$P\wedge\lnot Q$ is true ,then derive a logical contradiction .This
implies that $P\wedge\lnot Q$ is false . Hence $P\rightarrow Q$
is true by double negation equivalence.\end{description}

\end{document}
