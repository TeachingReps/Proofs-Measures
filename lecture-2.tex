%% LyX 2.1.3 created this file.  For more info, see http://www.lyx.org/.
\documentclass[a4paper,english,12pt]{article}
\usepackage{%
	amsfonts,%
	amsmath,%	
	amssymb,%
	amsthm,%
	bbm,%
	biblatex,%
	caption,%
	color,%
	enumerate,%
	epsfig,%
	epstopdf,%
	geometry,%
	graphicx,%
	hyperref,%
	latexsym,%
	mathtools,%
	multicol,%
	pgf,%
	%pgfplots%
	%pgfplotstable,%
	pgfpages,%
	proof,%
	psfrag,%
	subfigure,%	
	tikz,%
	ulem,%
	url%
}

\usepackage[mathscr]{eucal}
\usepgflibrary{shapes}
\usetikzlibrary{%
  arrows,%
	backgrounds,%
	chains,%
	decorations.pathmorphing,% /pgf/decoration/random steps | erste Graphik
	decorations.text,%
	matrix,%
  positioning,% wg. " of "
  fit,%
	patterns,%
  petri,%
	plotmarks,%
  scopes,%
	shadows,%
  shapes.misc,% wg. rounded rectangle
  shapes.arrows,%
	shapes.callouts,%
  shapes%
}

\theoremstyle{plain}
\newtheorem{thm}{Theorem}[section]
\newtheorem{lem}[thm]{Lemma}
\newtheorem{prop}[thm]{Proposition}
\newtheorem{cor}[thm]{Corollary}

\theoremstyle{definition}
\newtheorem{defn}[thm]{Definition}
\newtheorem{conj}[thm]{Conjecture}
\newtheorem{exmp}[thm]{Example}
\newtheorem{assum}[thm]{Assumptions}

\theoremstyle{remark}
\newtheorem{rem}{Remark}
\newtheorem{note}{Note}

\newcommand{\norm}[1]{\left\lVert#1\right\rVert}
\newcommand{\tr}{\operatorname{tr}}
\newcommand{\Real}{\mathbb{R}}

\makeatletter
\def\th@plain{%
  \thm@notefont{}% same as heading font
  \itshape % body font
}
\def\th@definition{%
  \thm@notefont{}% same as heading font
  \normalfont % body font
}
\makeatother
\date{}
%\usepackage[T1]{fontenc}
%\PassOptionsToPackage{normalem}{ulem}
%\usepackage{ulem}

%\makeatletter

%%%%%%%%%%%%%%%%%%%%%%%%%%%%%% LyX specific LaTeX commands.
%\pdfpageheight\paperheight
%\pdfpagewidth\paperwidth


%\makeatother

%\usepackage{babel}
\begin{document}

\title{Lecture 2: Strategies for Proofs}
\author{Parimal Parag}
\maketitle

\section{Introduction}
Thales of Miletus (6 BC) is credited with introducing the concepts of logical proof for abstract propositions. Euclid popularized the axiomatic system of proofs in his book \emph{The elements}, written over 2000 years ago. In this book, he proves a number of geometric theorems based on axioms.

We must ask why do we need proofs. There are four mains reasons for this endeavor. First, intuition is fallible (sometimes we see what we want to see). Therefore, proofs are necessary to infallibly ascertain facts. Second, we need proofs to explain why things are true. Not all proofs are instructive, however non-intuitive proofs are still better than no proofs. Third, writing proofs provide us a thorough understanding of the problem at hand. Last but not the least, proofs are useful to communicate ideas in language of mathematics. 

\begin{defn}[Mathematical Proof] A convincing argument that starts from the premises and logically deduces the desired conclusion.
\end{defn}
We usually prove theorems, propositions, lemmas, corollaries, and exercises. Theorems are important results, propositions are less important results. Lemmas are small independent statements that can be used to prove other results. Corollaries follow easily from other results, and exercise is something left for reader to verify. 

To prove theorems, we need existing theorems and definitions. Axioms are starting points for this chain.
\begin{defn}[Axioms] Facts about mathematical objects assumed without proof are called \textbf{axioms}.
\end{defn}
\begin{defn}[Axiomatic System] Body of knowledge that can be derived from a set of axioms is called \textbf{axiomatic system}.
\end{defn}
Sets are fundamental objects and considered basis for all arguments. Each branch of mathematics has specific st of axioms for associated objects. For example, algebra has axiomatically defined objects such as groups, rings, fields, etc.

\begin{exmp} Show that ``sum of even numbers are even.'' Precise statement for the above problem is `` if $n$ and $m$ are even integers, then $n+m$ is an even integer.''\\
Proving this statement precise definition of terms, such as integers where even and odd are defined. We also need understanding of standard properties such as closure under addition, subtraction, and multiplication, and distributive law of these properties over integers.
\end{exmp}
\begin{defn}Let $n$ be an integer. We say that $n$ is \textbf{even} if there is some integer $k$ such that $n=2k$. We say that $n$ is \textbf{odd} if there is some integer $j$ such that $n = 2j+1$.
\end{defn}
\begin{thm} Let $n$ and $m$ be integers.
\begin{enumerate}[i]
	\item \label{thm:even-even} If $n$ and $m$ are both even, then $n+m$ is even.
	\item \label{thm:odd-odd}If $n$ and $m$ are both odd, then $n+m$ is even.
	\item \label{thm:even-odd} If $n$ is even and $m$ is odd, then $n+m$ is odd.
\end{enumerate}
\end{thm}
\begin{proof}
We will prove part~\ref{thm:even-even}. Suppose $n$ and $m$ are both even, then there exist integers $k$ and $l$ such that $n=2k$ and $m=2l$. Therefore, $n+m=2(k+l)$ by distributive law of multiplication over additions on integers. Further, $(k+l)$ is also an integer by closure property of integer addition. Hence, $n+m$ is even. Parts~\ref{thm:odd-odd} and~\ref{thm:even-odd} can be shown similarly.
\end{proof}

Following are features of good proof. 
\begin{itemize}
\item Proofs should rely completely on the definitions.
\item A proof should be written in grammatically correct language.
\item A proof uses rules of inference implicitly.
\end{itemize}

\subsection{Types of Proofs }
\begin{description}
\item [{Direct~Proofs~$P\rightarrow Q$}] Assume that $P$ is true and
produce a series of steps eventually leading to $Q$.
\item [{Proof~by~contrapositive~$\lnot Q\rightarrow\lnot P$}] Assume
$Q$ is false and produce a series of steps eventually leading to
$\lnot P$.
\item [{Proof~by~contradiction~$\lnot(P\rightarrow Q)\leftrightarrow P\wedge\lnot Q$}] Assume
$P\wedge\lnot Q$ is true ,then derive a logical contradiction .This
implies that $P\wedge\lnot Q$ is false . Hence $P\rightarrow Q$
is true by double negation equivalence.\end{description}


\begin{defn} Let $x$ be a real number.We say $x$ is a rational number
if there exists integers $m$ and $n$ s.t $m\ne0$ and $x=\frac{n}{m}$
and there are no 2 common factors between $n$ and $m$ other than
$1$ or $-1$.
\end{defn}

\begin{thm} Prove that $\sqrt{2}$ is an irrational number.
\end{thm}

 
\begin{proof} Let $x$ be a rational number s.t $x=\sqrt{2}$.Then $x^{2}=2$.So
by definition above $\frac{n^{2}}{m^{2}}=2$ or $n^{2}=2m^{2}$.So
$n^{2}$ is an even number , therefore $n$ must be even . Hence there
exists an integer $k$s.t $n=2k$. Thus we have $4k^{2}=2m^{2}$ i.e
$2k^{2}=m^{2}$.Then $m$ must also be even i.e $m$ and $n$ have
$2$ as common factor which contradicts the fact that both must have
no common factor other than $1$ or $-1$ by definition above .Hence
$x=\sqrt{2}$ is an irrational number .
\end{proof}


\end{document}
