\documentclass[a4paper,english,12pt]{article}
\usepackage{%
	amsmath,%
	amsfonts,%
	amssymb,%
	amsthm,%
	hyperref,%
	url,%
	latexsym,%
	epsfig,%
	graphicx,%
	psfrag,%
	subfigure,%	
	color,%
	tikz,%
	pgf,%
	pgfplots,%
	pgfplotstable,%
	pgfpages,%
	proofs%
}

\usepgflibrary{shapes}
\usetikzlibrary{%
  arrows,%
	backgrounds,%
	chains,%
	decorations.pathmorphing,% /pgf/decoration/random steps | erste Graphik
	decorations.text,%
	matrix,%
  positioning,% wg. " of "
  fit,%
	patterns,%
  petri,%
	plotmarks,%
  scopes,%
	shadows,%
  shapes.misc,% wg. rounded rectangle
  shapes.arrows,%
	shapes.callouts,%
  shapes%
}

\theoremstyle{plain}
\newtheorem{thm}{Theorem}[section]
\newtheorem{lem}[thm]{Lemma}
\newtheorem{prop}[thm]{Proposition}
\newtheorem{cor}[thm]{Corollary}

\theoremstyle{definition}
\newtheorem{defn}[thm]{Definition}
\newtheorem{conj}[thm]{Conjecture}
\newtheorem{exmp}[thm]{Example}
\newtheorem{assum}[thm]{Assumptions}

%\theoremstyle{remark}
\newtheorem{rem}{Remark}
\newtheorem{note}{Note}

\makeatletter
\def\th@plain{%
  \thm@notefont{}% same as heading font
  \itshape % body font
}
\def\th@definition{%
  \thm@notefont{}% same as heading font
  \normalfont % body font
}
\makeatother
\date{}

\title{Lecture 7: Relations}
\author{}

\begin{document}
\maketitle

\section{Relation}
Let $A$ and $B$ be sets. A relation $R$ from $A$ to $B$ is a subset of Cartesian product $R \subseteq A \times B $. If $a \in A$ and $b \in B$, we write $a R b$ if $(a,b) \in R$ and $a R b$ if $(a,b) \notin R$. 
A relation from $A$ to $A$ is called a relation on $A$.
\begin{exmp}
\begin{enumerate}
1. Let $A=\{1,2,3\}$ and $B =\{x,y,z\}$ and S$\subseteq$ \{(1,y),(1,z),(2,y)\} then $1Sy, 1Sz, 2Sy$\\\\
2. P=\{set of all people\}\\
R$\subseteq$ P$\times$P s.t R=\{(x,y)$\in$ P$\times$ P:x and y have at least one parent common\}\\
3. P=\{set of all people\} , B =\{set of all books\}\\
Relation T$\subseteq$ P$\times$B such that \\
T=\{(x,y)$\in$ P$\times B$: x has read y \}\\
4. Let A be a set define a relation on P(A), by saying that P,Q $\in$ P(A) are related iff P$\subseteq$Q\\
U =\{(P,Q)$\in$ P(A)$\times$P(A):P$\subseteq$Q\}\\\\
5. A function f: A$\rightarrow$ B is subset of A$\times$B satisfying certain conditions.Hence function f: A$\rightarrow$B are also relation from A$\rightarrow$B.
\end{enumerate}
\end{exmp}
\textbf{Notice the difference between Functions and relations}\\
$\bullet$ Domain doesn't have to be covered\\
$\bullet$ More than one element for each a$\in$A\\\\
\textbf{Definition}\\
Let A and B be non empty sets and let R be a relation from A to B.For each element x$\in$A, define the "Relation Class" of x with respect to relation R, denoted R[x], to be the set R[x] =\{y$\in$B : xRy\} if the relation R is understand from the context, we will often write [x] instead of R[x]\\
\textbf{Examples}\\
consider the previous examples \\
1.For this relation we see that [1]=\{y,z\}, and [2]=\{y\}, [3]=$\emptyset$\\\\
2.There are a number of distinct cases here, and we will examine a few of them.
If x is the only child of each of her parents, then [x] = \{x\}, where we observe that
x has the same parents as herself. If y and z are the only two children of each of
their parents, then [y] = \{y, z\} = [z]. If a has one half-sibling b by her father, and
another half-sibling c by her mother, and each of b and c has no other siblings or
half-siblings, then [a] = \{a, b, c\}, and [b] = \{a, b\}, and [c] = \{a, c\}.\\\\
3.For the relation $<$, we see that [x] = (x, $\infty$) for all x $\in$ $\mathbb{R}$, and for the relation
$\leq$, we see that [x] = [x, $\infty$) for all x $\in$ $\mathbb{R}$.\\\\
\textbf{Definition}\\
Let A$\neq \emptyset$ and let R be a relation on A\\
1."Reflexive" if xRx for all x$\in$A\\\\
2."symmetric" if xRy implies yRx for all x$\in$A\\\\
3."transitive" if xRy and yRz implies xRz for all x,y,z$\in$A\\\\
\textbf{Examples}\\\\
1. Congruence of triangles is "Reflexive","symmetric",Transitive".\\\\
2. The relation "$\leq$" is "reflexive",but not "Transitive ,not "Symmetric".\\\\
3. C=\{1,2,3\} P=\{(2,2),(3,3),(2,3),(3,2)\}\\\\
  Not "Reflexive" as (1,1)$\notin$P\\\\
  Its "symmetric" as \{(2,3),(3,2)\}$\in$P\\\\
  Its "Transitive" as \{(2,3),(3,2),(2,2)\}$\in$P\\\\
4. B=\{x,y,z\}       T=\{(x,x),(y,y),(z,z),(x,y),(y,z)\}\\\\
 Its "Reflexive" as           \{(x,x),(y,y),(z,z)\}$\in$T\\\\
 But not "symmetric" as\{(y,x),(z,y)\}$\notin$T\\\\
 Not "transitive" as(x,z)$\notin$T\\\\
5. Relation of one person being cousin of    another is "symmetric", not "Reflexive", not "Transitive"\\\\
6. < relation is not "Reflexive", not "Symmetric" but its "Transitive".\\\\
7. Relation of one person being daughter of another person :none.\\\\
\section{Equivalence Relation}
\textbf{Definition}\\\\
Let A be a set and let $\sim$ be a relation on A.The relation $\sim$ is an "equivalence relation" if it is reflexive, symmetric, transitive.\\\\
\textbf{Examples}\\\\
1. $\sim$ : same age on set of people.\\\\
2. $\sim$ : on $\mathbb{R}$\\\\
\textbf{Definition}\\\\
Let A be a non empty set and $\sim$ be an equivalence relation on A .The relation classes of A w.r.t $\sim$ are called "equivalence classes".The "quotient set" of A define and equivalence relation $\sim$ is the set of all equivalence classes of A w.r.t $\sim$ that is the set \{[x]:x$\in$A\}\\\\
The quotient set of A and $\sim$ is denoted by A/$\sim$\\\\
\textbf{Example}
1. Let P be the set of all people, and let ∼ be the relation on P defined
by x $\sim$ y if and only if x and y are the same age (in years). If person x is 19 years
old, then the equivalence class of x is the set of all 19-year olds. Each element of the
quotient set P/$\sim$ is itself a set, where there is one such set consisting of all 1-year-
olds, another consisting of all 2-year olds, and so on. Although there are billions of
people in P, there are fewer than 125 elements in P/$\sim$, because no currently living
person has reached the age of 125.\\\\
2. Let $\sim$be the relation on R 2 defined by (x, y) $\sim$ (z, w) if and only if y$-$x =
w$-$z, for all (x, y), (z, w) $\in$ $\mathbb{R}$$^2$ . It can be verified that $\sim$ is an equivalence relation. We
want to describe the partition $\phi(\sim)$ of $\mathbb{R}$$^2$. Let (x, y) $\in$ $\mathbb{R}$$^2$ . Then [(x, y)] = {(z, w) $\in$
$\mathbb{R}$$^2$ | w − z = y − x}. Let c = y − x. Then [(x, y)] = {(z, w) $\in$ R 2 | w = z + c}, which is
just a line in $\mathbb{R}$$^2$ with slope 1 and y-intercept c. Hence $\phi(\sim)$ is the collection of all
lines in $\mathbb{R}$$^2$ with slope 1.\\\\
2. Let C = \{[n, n + 1)\} Where n$\in$Z Then C is a partition of $\mathbb{R}$\\
Y(C).= \{(x,y)$\in$ $\mathbb{R}\times\mathbb{R}$| (x,y)$\in$[n,n$+$1)\}
  







\end{document}
