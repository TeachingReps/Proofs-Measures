\documentclass[a4paper,english,12pt]{article}
\usepackage{%
	amsmath,%
	amsfonts,%
	amssymb,%
	amsthm,%
	hyperref,%
	url,%
	latexsym,%
	epsfig,%
	graphicx,%
	psfrag,%
	subfigure,%	
	color,%
	tikz,%
	pgf,%
	pgfplots,%
	pgfplotstable,%
	pgfpages,%
	proofs%
}

\usepgflibrary{shapes}
\usetikzlibrary{%
  arrows,%
	backgrounds,%
	chains,%
	decorations.pathmorphing,% /pgf/decoration/random steps | erste Graphik
	decorations.text,%
	matrix,%
  positioning,% wg. " of "
  fit,%
	patterns,%
  petri,%
	plotmarks,%
  scopes,%
	shadows,%
  shapes.misc,% wg. rounded rectangle
  shapes.arrows,%
	shapes.callouts,%
  shapes%
}

\theoremstyle{plain}
\newtheorem{thm}{Theorem}[section]
\newtheorem{lem}[thm]{Lemma}
\newtheorem{prop}[thm]{Proposition}
\newtheorem{cor}[thm]{Corollary}

\theoremstyle{definition}
\newtheorem{defn}[thm]{Definition}
\newtheorem{conj}[thm]{Conjecture}
\newtheorem{exmp}[thm]{Example}
\newtheorem{assum}[thm]{Assumptions}

%\theoremstyle{remark}
\newtheorem{rem}{Remark}
\newtheorem{note}{Note}

\makeatletter
\def\th@plain{%
  \thm@notefont{}% same as heading font
  \itshape % body font
}
\def\th@definition{%
  \thm@notefont{}% same as heading font
  \normalfont % body font
}
\makeatother
\date{}

\title{Lecture 7: Relations}
\author{}

\begin{document}
\maketitle

\section{Relation}
Relation between two objects signify some connection between them. For example, relation of one person being biological parent of another. If we take any two people at random, say persons $X$ and $Y$, then either $X$ is a parent of $Y$ or not. This depends on our knowledge of knowing whether $(X,Y)$ is a parent child pair. Alternatively, we can list all pairs of people $(X,Y)$ such that $X$ is a parent of $Y$. Then, knowing whether $X$ and $Y$ are related is equivalent to checking whether pair $(X,Y)$  belongs to this set of listed pairs. We formalize this notion below.
\begin{defn}[Relation] Let $A$ and $B$ be sets. A \textbf{relation} $R$ from $A$ to $B$ is a subset
\begin{equation*}
R \subseteq A \times B. 
\end{equation*}
If $a \in A$ and $b \in B$, we write $aRb$ if $(a,b) \in R$ and $a \centernot R b$ if $(a,b) \notin R$. A relation from $A$ to $A$ is called a relation on $A$.
\end{defn}
\begin{exmp} Following are some examples of relations.
\begin{enumerate}
\item Let $A=\{1,2,3\}$ and $B =\{x,y,z\}$. Then, we can define a relation $S = \{(1,y),(1,z),(2,y)\}$ such that $1Sy, 1Sz, 2Sy$.
\item Let $P$ be set of all people. Then, we define a relation $R$ as $R \subseteq P \times P$ such that $R =\{(x,y) \in P \times P :x \text{ and } y \text{ have at least one parent in common}\}$.
\item Symbols $<$ and $\leq$ represent relations on $\mathbb{R}$.
\item Let P be set of all people, and B be set of all books. Then, we can define a relation $T \subseteq P \times B$ such that $T=\{(x,y) \in P \times B: x \text{ has read } y \}$. 
\item Let $A$ be a set. Define a relation on $\mathcal{P}(A)$, by saying that $P,Q \in \mathcal{P}(A)$ are related iff $P \subseteq Q$. Let $U =\{(P,Q) \in \mathcal{P}(A) \times \mathcal{P}(A) : P \subseteq Q\}$.
\item A function $f: A \to B$ is subset of $A \times B$ satisfying certain conditions. Hence, a function $f: A \to B$ is also a relation from $A$ to $B$.
\end{enumerate}
\end{exmp}
\begin{rem} Relations are more general than functions.  6 6grNotice the difference between functions and relations.
\begin{enumerate}
	\item Possible to have no ordered pair for some $a \in A$.
	\item Possible to have more than one ordered pair for some $a \in A$.
\end{enumerate}
\end{rem}
\begin{defn}[Relation Class] Let $A$ and $B$ be non-empty sets and let $R$ be a relation from $A$ to $B$.For each element $x \in A$, define the \textbf{relation class} of $x$ with respect to relation $R$, denoted $R[x]$, to be the set
\begin{equation*} 
R[x] =\{y \in B : x R y \}.
\end{equation*} 
If the relation $R$ is understood from the context, we will often write $[x]$ instead of $R[x]$.
\end{defn}
\begin{exmp} Let's consider the previous examples.
\begin{enumerate}
	\item For this relation we see that $[1]=\{y,z\}$, $[2]=\{y\}$, and $[3] = \emptyset$. We have an example of relation class that is empty.
	\item There are a number of distinct cases here, and we will examine a few of them. 
	\begin{enumerate}
		\item If $x$ is the only child of each of her parents, then $[x] = \{x\}$, where we observe that
$x$ has the same parents as herself. Notice that for any two distinct single children, $[x] \cup [y] = \emptyset$. That is, we have examples of relations classes that are disjoint.
		\item If $y$ and $z$ are the only two children of each of their parents, then $[y] = \{y, z\} = [z]$. 
		\item If $a$ has one half-sibling $b$ by her father, and another half-sibling $c$ by her mother, and each of $b$ and $c$ have no other siblings or half-siblings, then $[a] = \{a, b, c\}$, and $[b] = \{a, b\}$, and $[c] = \{a, c\}$.
	\end{enumerate}
	\item For the relation $<$, we see that $[x] = (x, \infty)$ for all $x \in \mathbb{R}$, and for the relation
$\leq$, we see that $[x] = [x, \infty)$ for all $x \in \mathbb{R}$.
	\item Relation class of a person $x$ is $[x] = \{\text{ set of books read by }x\}$. 
	\item Relation class of a subset $P \subseteq A$ is $[P] = \{ Q \subset A: p \subseteq Q \}$.
	\item Relation class of a function $x$ is $[x] = \{f(x)\}$.
\end{enumerate}
\end{exmp}
We see that relation classes can be empty. Two relation classes can be disjoint or have non-empty intersection. This is due to the fact that relations are very general and have no structures. In the following we would add some structure to define two specific relations.
\begin{defn} Let $A \neq \emptyset$ and let $R$ be a relation on $A$.
	\begin{enumerate}
		\item Relation $R$ is \textbf{reflexive} if $xRx$ for all $x \in A$.
		\item Relation $R$ is \textbf{non-reflexive} if $xRx$ doesn't hold for any $x \in A$.
		\item Relation $R$ is \textbf{symmetric} if $xRy$ implies $yRx$ for all $x \in A$.
		\item Relation $R$ is \textbf{transitive} if $xRy$ and $yRz$ implies $xRz$ for all $x,y,z \in A$.
		\item Relation $R$ is \textbf{comparable} if $x \neq y$, then either $xRy$ or $yRx$ for all $x,y \in A$.
	\end{enumerate}
\end{defn}
\begin{exmp} We look at some relations and which of the above-defined properties they satisfy. 
\begin{enumerate}
	\item Congruence of triangles is reflexive, symmetric, and transitive.
	\item Relation of one person weighing within $5$ kgs of another person is reflexive, symmetric, but not transitive.
	\item The relation $\leq$ on real numbers is not symmetric. However, this relation is reflexive, and transitive. 
	\item Let $C=\{1,2,3\}$, and a relation $P$ on set $C$ such that $P=\{(2,2),(3,3),(2,3),(3,2)\}$. Then $P$ is not reflexive as $(1,1) \notin P$, symmetric as $\{(2,3),(3,2)\} \in P$, and transitive as $\{(2,3),(3,2),(2,2)\} \in P$.
	\item Let $B=\{x,y,z\}$ and a relation $T$ on set $B$ such that $T = \{(x,x),(y,y),(z,z),(x,y),(y,z)\}$. Relation $T$ is reflexive as $\{(x,x),(y,y),(z,z)\} \in T$, not symmetric as $\{(y,x),(z,y)\} \notin T$, and not transitive as $(x,z) \notin T$.
	\item Relation of one person being cousin of another is symmetric, but not reflexive, or transitive. In fact it is anti-reflexive.
	\item Relation $<$ on real numbers is neither reflexive, nor symmetric. However, it is anti-reflexive, comparable, and transitive.
	\item Relation of one person being daughter of another person is neither reflexive, nor symmetric, nor transitive.
\end{enumerate}
\end{exmp}

\end{document}
