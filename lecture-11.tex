\documentclass[a4paper,english,12pt]{article}   	% use "amsart" instead of "article" for AMSLaTeX format
\usepackage{%
	amsmath,%
	amsfonts,%
	amssymb,%
	amsthm,%
	hyperref,%
	url,%
	latexsym,%
	epsfig,%
	graphicx,%
	psfrag,%
	subfigure,%	
	color,%
	tikz,%
	pgf,%
	pgfplots,%
	pgfplotstable,%
	pgfpages,%
	proofs%
}

\usepgflibrary{shapes}
\usetikzlibrary{%
  arrows,%
	backgrounds,%
	chains,%
	decorations.pathmorphing,% /pgf/decoration/random steps | erste Graphik
	decorations.text,%
	matrix,%
  positioning,% wg. " of "
  fit,%
	patterns,%
  petri,%
	plotmarks,%
  scopes,%
	shadows,%
  shapes.misc,% wg. rounded rectangle
  shapes.arrows,%
	shapes.callouts,%
  shapes%
}

\theoremstyle{plain}
\newtheorem{thm}{Theorem}[section]
\newtheorem{lem}[thm]{Lemma}
\newtheorem{prop}[thm]{Proposition}
\newtheorem{cor}[thm]{Corollary}

\theoremstyle{definition}
\newtheorem{defn}[thm]{Definition}
\newtheorem{conj}[thm]{Conjecture}
\newtheorem{exmp}[thm]{Example}
\newtheorem{assum}[thm]{Assumptions}

%\theoremstyle{remark}
\newtheorem{rem}{Remark}
\newtheorem{note}{Note}

\makeatletter
\def\th@plain{%
  \thm@notefont{}% same as heading font
  \itshape % body font
}
\def\th@definition{%
  \thm@notefont{}% same as heading font
  \normalfont % body font
}
\makeatother
\date{}


\title{Lecture 11 : Countability}
\author{}

\begin{document}
\maketitle

\section{Countability}
\begin{defn} Let $n \in \N$, Let $[n]$ denote the set $\{1,2,3,....n\}$
\begin{enumerate}
\item A set $\textbf{A}$ is finite if it is either empty or $A \sim [n]$ for some $n \in \N$.
\item The set $\textbf{A}$ is infinite if it is not finite.
\item The set $\textbf{A}$ is countably infinite if $A \sim \N$.
\item The set $\textbf{A}$ is countable if it is finite or countably infinite.
\item The set $\textbf{A}$ is uncountable if it is not countable.
\end{enumerate}
\end{defn}
\begin{lem} 
\begin{enumerate}
\item The set $\N$ is infinite.
\item A countably infinite set if $A \sim \N$.
\end{enumerate}
\end{lem}
\begin{proof} $\textbf{(i)}$ Suppose that $\N$ is finite. Because $\N \neq \emptyset$ then $\exists n \in \N$ such that $\N \sim [n]$.
Let $f \colon [n] \to \N$ be bijective function. From previous theorem it follows that $\exists k \in [n]$ such that $f(k) \geq f(i) $ $\forall i \in [n]$. Therefore 
$f(k)+1 \notin f([n])$. But $f(k)+1 \in \N$,Therefore $f$ is not Surjective. This is contradiction since $f$ is bijective. Hence The set $\N$ is infinite.\\
$\textbf{(ii)}$ Let B be a countably infinite set, Then $B \sim \N$ by definition. Suppose that B is finite, then it means  $\N$ is finite. Which is a contradiction to part 1 of this lemma.
\end{proof} 
\begin{exmp} Let A={1,2} then $P(A)=\{ \emptyset,\{1\},\{2\},\{1,2\}\}$. Therefore $A \nsim P(A) $.
\end{exmp}
\begin{thm} Let A be a set then $A \nsim P(A) $.
\end{thm}
\begin{proof} Consider $A = \emptyset$ then $P(A) = \{ \emptyset \}$ , Hence there can not be a bijective function $P(A) \to A$. So $A \nsim P(A) $.\\
Suppose $A \neq \emptyset$ , Suppose $A \sim P(A)$ then $\exists$ bijection $f \colon A \to P(A)$.\\
Let $D=\{a \in A  | a \notin f(a)\}$. Notice $D \subseteq A$. Hence $D \in P(A)$. Since $f$ is surjective we can choose $d \in A$ such that $f(d)=D$.\\
Is $d \in D$?\\
Suppose $d \in D$, then $d \notin f(d)=D$.\\
Suppose $d \notin D$, then $d \in f(d)=D$.\\
We therefore have a contradiction. So $A \nsim P(A)$.
\end{proof}
\begin{cor} The set P($\N$) is uncountable.
\end{cor}
\begin{defn} Let A and B be sets. we say that $A \leq B$ if there is an injection function $f \colon A \to B$. we say that $A < B$ if $A \leq B$ and $A \nsim B$.
\end{defn}
It is easy to see that for any set A their is an injective function from $A \to P(A)$ and hence $A \leq P(A)$. By above Theorem $A \nsim P(A)$ so $A < P(A)$. Applying this fact to set $\N$ repeatedly we get following
\[ \N < P(\N) < P(P(\N)) < P(P(P(\N)))......\]
Because all the sets in this sequence other than the first are uncountable, we therefore see that there are infinitely many different cardinalities among the uncountable sets.The cardinality of $\N$ is denoted by $\aleph _{0}$ and called "cardinal number" even though it is not real number. it is common to denote the cardinality of $P(\N)$ by  $ 2 ^{\aleph _{0}}$



\begin{thm}[Schroeder - Bernstien theorem]
Let A and B be sets. Suppose that $A \leq B$ and $B \leq A$, then $A \sim B$.
\end{thm}
\begin{proof}
Let $f \colon A \to B$ and $g \colon B \to A$ be injective functions.\\
Let T be subset of A constructed as follows\\
Let $T_{0}=A \backslash g(B)$.
$T_{n+1}= (g \circ f)(T_{n})$ $\forall n \in \N_{0}$.\\
$T=\bigcup_{n \in \N_{0}} T_{n}$.\\
We need to define a bijection $h \colon A \to B$.\\
\[ h(x)\, \, = f(x)  \quad  x \in T \] 
$\qquad \qquad \qquad \qquad \qquad \qquad \qquad \qquad \, \,=g^{-1}(x) \quad  x \in A \backslash T$.
\begin{figure}[hhhh]
\centering
\scalebox{.8}{
\begin{tikzpicture}
  \matrix (m) [matrix of math nodes,row sep=3em,column sep=4em,minimum width=2em]
  {
     N & N \\
     P(A) & P(A) \\};
  \path[-stealth]
    (m-1-1) edge node [left] {T} (m-2-1)
            edge node [below] {S} (m-1-2)
    (m-2-1) edge node [below] {$ g \circ f$} (m-2-2)
     (m-1-2) edge node [right] {T} (m-2-2);
 \end{tikzpicture}}
\caption{}
\label{Fig1}
\end{figure}
\\$T=\bigcup_{n \in \N_{0}} T_{n}$.\\
$(g \circ f)(T)=\bigcup_{n \in \N_{0}} (g \circ f)(T_{n})$
$= \bigcup _{n \in \N} T_{n} \subseteq T$.
\\To show $h$ is bijection\\
\textbf{Injectiviity:} Since $h \arrowvert _{T}$ and $h \arrowvert _{A \backslash T}$ are injective and if $x_{1} \in T$ and $x_{2} \in A \backslash T$ If $h(x_{1})=h(x_{2})$ implies $f(x_{1})=g^{-1}(x_{2})$ or $(g \circ f)(x_{1})=x_{2}$, that $x_{2} \in (g \circ f)(T) \subseteq T$ from above derivation. which is contradiction since  $x_{2} \in A \backslash T$. so $h(x_{1}) \neq h(x_{2})$. Hence $h$ is injective.\\
\textbf{Surjectivity:} 
Since $h$ is injective $f(T)$ and $g^{-1}(A \backslash T)$ sets  are disjoint. 
And $f(T) \cup g^{-1}(A \backslash T) \subseteq B$.\,\,\,\,\,\,$ g^{-1}(A \backslash T) \subseteq B$.
To show $h$ is surjective It suffices to show $B \backslash f(T) \subseteq g^{-1}(A \backslash T)$. Because we need to show $\forall b \in B$  $\exists a \in A$ such that $f(a)=b$.\,\,$\forall b \in f(T) \subseteq B$ this property is satisfied. Need to show for $\forall b \in B \backslash f(T)$.
If we can show sets $B \backslash f(T)$ and $g^{-1}(A \backslash T)$ are equal from injectivity of $h$ and property of functions we can satisfy surjectivity.\\
Let $y \in B \backslash f(T) = \{ y \in B |\, \,  y \neq f(x)\,\,   for \,\,  any \,\,  x \in T\}$. Now $g(y) \neq (g \circ f)(x)$. For any $x \in T$, Since $g$ is injective. That is $g(y) \neq (g \circ f)(T) \subseteq T$(followed from above given derivation)  OR $g(y) \in A \backslash T$. That is $y \in g^{-1}(A \backslash T)$.

\end{proof}

\begin{exmp} Let $a<b \in \mathbb{R}$. We will use Schroeder - Bernstien theorem to show $[a,b] \sim (a,b)$. Suffices to show $[-1,1] \sim (-1,1)$.\\
Let $f \colon (-1,1) \to [-1,1]$ be defined by $f(x)=x$ $\forall x \in (-1,1)$ it is easy to see $f$ is injective.\\
Let $g \colon [-1,1] \to (-1,1)$ be defined by $g(x)=x/2$ $\forall x \in [-1,1]$.
\end{exmp}
\begin{thm}[Trichotomy law for sets] Let A and B be sets then $A \leq B $ or $B \leq A$.
\end{thm}

\section{Finite sets and countable sets}
\begin{defn} Let A be a set. Suppose A is finite. The "cardinality" of denoted by $|A|$, is defined as follows. If $A= \emptyset$ then $|A|=0$, If $A \neq \emptyset$, then $|A|=n$ where $A \sim [n]$.
\end{defn}
\begin{lem} Let $n,m \in \N$ then $[n] \sim [m]$ iff n=m.
\end{lem}
\begin{cor} Let A,B be finite sets then $A \sim B$ iff $|A|=|B|$.
\end{cor}
\begin{thm} Let A be a set. Suppose A is finite
\begin{enumerate}
\item If $X \subseteq A$, then X is finite.
\item If $X \subseteq A$, then $|A|=|X|+|A\backslash X|$.
\item If $X \subset A$, then $|X|<|A|$.
\item If $X \subset A$, then $X \nsim A$.
\end{enumerate}
\end{thm}
\begin{proof}
\textbf{(i)} It follows immediately from the theorem Let $S \subseteq \{1,2,3,...,n\}$  be a non-empty subset. Then there is a bijective function $g \colon \{1,2,3,...,n\} \to \{1,2,3,...,n\}$ such that $g(S) =  \{1,2,3,...,k\}$ for some $k \in \N$ and $k \leq n$.If S is a proper subset of $\{1,2,3,...,n\}$ then $r < n$.\\
\textbf{(ii)} If $A - X$ is $\emptyset$ result is trivial. If $A - X \neq \emptyset$ and let $|A|=n$ and n>0. Let $f \colon A \to \{1,2,3,...,n\}$ be bijection. $f(X) \subseteq \{1,2,3,...,n\}$ for $X \subseteq A$. We can find bijection $g \colon \{1,2,3,...,n\} \to \{1,2,3,...,n\}$ such that $g(f(X)) =  \{1,2,3,...,k\}$ for some $k \in \N$ and $k \leq n$. We can se that $g \circ f$ is bijection. Then it follows that $X \sim  \{1,2,3,...,k\}$ that is $|X|=k$.\\
\[ (g \circ f)(A-X)=(g \circ f)(A)-(g \circ f)(X)= g(f(A))-g(f(X))\]
$\qquad \qquad \qquad \qquad \qquad \,\,\,\,\,\,\,\,   =\{1,2,3,...,n\}-\{1,2,3,...,k\} = \{k+1,k+2,...,n\}$\\
Since $g \circ f$ is bijection it follows that $|A-X| \sim \{k+1,k+2,...,n\}$. We can find a bijection from $\{1,2,3,...,n-k\} \to \{k+1,k+2,...,n\}$. Hence $|A-X|=n-k$.
Which completes the proof.\\
\textbf{(iii)} It follows from the above part (ii) proof.\\
\textbf{(iv)} This part of the theorem follows from Part (iii) and Corollary.

\end{proof}
\begin{cor} Let A be a set, then A is infinite iff it contains an infinite subset.
\end{cor}




\end{document}  