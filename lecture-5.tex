%% LyX 2.1.3 created this file.  For more info, see http://www.lyx.org/.
\documentclass[a4paper,english,12pt]{article}
\usepackage{%
	amsfonts,%
	amsmath,%	
	amssymb,%
	amsthm,%
	bbm,%
	biblatex,%
	caption,%
	color,%
	enumerate,%
	epsfig,%
	epstopdf,%
	geometry,%
	graphicx,%
	hyperref,%
	latexsym,%
	mathtools,%
	multicol,%
	pgf,%
	%pgfplots%
	%pgfplotstable,%
	pgfpages,%
	proof,%
	psfrag,%
	subfigure,%	
	tikz,%
	ulem,%
	url%
}

\usepackage[mathscr]{eucal}
\usepgflibrary{shapes}
\usetikzlibrary{%
  arrows,%
	backgrounds,%
	chains,%
	decorations.pathmorphing,% /pgf/decoration/random steps | erste Graphik
	decorations.text,%
	matrix,%
  positioning,% wg. " of "
  fit,%
	patterns,%
  petri,%
	plotmarks,%
  scopes,%
	shadows,%
  shapes.misc,% wg. rounded rectangle
  shapes.arrows,%
	shapes.callouts,%
  shapes%
}

\theoremstyle{plain}
\newtheorem{thm}{Theorem}[section]
\newtheorem{lem}[thm]{Lemma}
\newtheorem{prop}[thm]{Proposition}
\newtheorem{cor}[thm]{Corollary}

\theoremstyle{definition}
\newtheorem{defn}[thm]{Definition}
\newtheorem{conj}[thm]{Conjecture}
\newtheorem{exmp}[thm]{Example}
\newtheorem{assum}[thm]{Assumptions}

\theoremstyle{remark}
\newtheorem{rem}{Remark}
\newtheorem{note}{Note}

\newcommand{\norm}[1]{\left\lVert#1\right\rVert}
\newcommand{\tr}{\operatorname{tr}}
\newcommand{\Real}{\mathbb{R}}

\makeatletter
\def\th@plain{%
  \thm@notefont{}% same as heading font
  \itshape % body font
}
\def\th@definition{%
  \thm@notefont{}% same as heading font
  \normalfont % body font
}
\makeatother
\date{}
%\usepackage[T1]{fontenc}
%\PassOptionsToPackage{normalem}{ulem}
%\usepackage{ulem}
%\usepackage{caption}
%\makeatletter
%\usepackage{multicol}
%%%%%%%%%%%%%%%%%%%%%%%%%%%%%% LyX specific LaTeX commands.
%\pdfpageheight\paperheight
%\pdfpagewidth\paperwidth


%\makeatother

%\usepackage{babel}
\begin{document}

\title{Lecture 5: Functions}
\author{}
\maketitle

\section{Functions}
We have all seen some form of functions in high school. For example, we have seen polynomial, exponential, logarithmic, trigonometric functions in calculus. These functions map real numbers to real numbers. We have also seen functions that map functions to functions, such a derivatives. Functions are of interest in many branches of mathematics, including enumerative combinatorics, topology, and group theory among others. An abstract understanding of function would be, an output $f(x)$ for each input $x$. We formalize this notion below.
\begin{defn} [Function]
 Let $A$ and $B$ be sets. A \textbf{function} (also called a \textbf{map}) from $A$ to $B$ denoted $f: A \to B$ is a subset $F \subseteq A \times B$ such that for each $a \in A$, there is a one and only one pair of the form $(a,b)$ in $F$. The set $A$ is called the domain of $f$ and the set $B$ is called the co-domain of $f$.
\end{defn}

\begin{rem} To show the equality of functions, we need to show that the domain, co-domain, and subset of the product of domain and co-domain satisfying the above conditions must agree.
\end{rem}

\begin{defn}
Let $A$ and $B$ be sets, and let $S \subseteq A$ be a subset. 
\begin{enumerate}
 \item  A \textbf{constant map} $f: A \to B$ is any function of the form $f(x) = b$ for all $x \in A$, where $b \in B$ is some fixed element.
 
 \item  The \textbf{identity map} on $A$ is the function $I_{A}: A \to A$ defined by $\mathbb{1}_{A}(x) = x$ for all $x \in A$.

 \item  The \textbf{inclusion map} from $S to A$ is the function $j: S \to A$ defined by $j(x) = x$ for all $x \in A$.

 \item If $f: A \to B$ is a map, the \textbf{restriction} of $f \to S$, denoted by $f |_{S}$ is the map $f|_{S}: S \to B$, defined by $f|_{S}(x)= x$ for all 
 $x \in S$.

 \item  If $g: A \to B$ is a map, an \textbf{extension} of $g$ to $A$ is any map $G: A \to B$ such that $G|_{s} = g$.

 \item   The \textbf{projection} map from $A \times B$ are the functions, $\pi_{1}: A \times B \to A$ and $\pi_{2}: A \times B \to B$ defined by 
 $\pi_{1}(a, b) = a$ and $\pi_{1}(a, b) = b$ for all $(a,b) \in A \times B$. Projection maps $\pi_{i}: A_{1} \times A_{2} \dots A_{p} = A_{i}$ for 
 any finite collection of sets $A_{1}, A_{2} \dots A_{p}$ are defined similarly.

\end{enumerate}
\end{defn}


\begin{defn} [Image and inverse image]
 Let $f: A \to B$ be a function.
 \begin{enumerate} [i)]
  \item For each $P \subseteq A$, let $f_{*}(P)$ be defined as $f_{*}(P) = \{ b \in B | b = f(P) \text{ for some } p \in P \} = \{ f(p) | p \in P\}$ is called 
  the image of $P$ under $f$.
  \emph{Remark}. The range (image) of $f$ is the set $f_{*}(A)$.
  \item For each $G \subseteq B$, let $f^{*}(Q)$ be defined as $f^{*}(Q) = \{ a \in A | f(a) = q \text{ for some } q \in Q\} \{ a \in A | f(a) \in Q \}$ is called 
  the inverse image of $Q$ under $f$.
 \end{enumerate}
\end{defn}

\section{Image and Inverse Image}
Let $A$ and $B$ be sets and $f:A\rightarrow B$ be a function. 
\begin{defn}[Image] For every $P\subseteq A$, the \textbf{image} of $P$ under $f$ is defined as follows:
\begin{equation*}
f_*(P)  =\{b\in B: b=f(p) \textnormal{ for some } p\in P\} =\{f(p):p\in P\}
\end{equation*}
\end{defn}

\begin{defn}[Inverse Image] For every $Q\subseteq B$, the \textbf{inverse image} (or \textbf{preimage}) of $Q$ under $f$ is defined as follows:
\begin{equation*}
f^*(Q)  =\{a\in A: f(a)=q \textnormal{ for some } q\in Q\} =\{a\in A:f(a)\in Q\}
\end{equation*}
\end{defn}

\begin{figure}[h]
\centering
\includegraphics[scale=0.6]{Figures/l5f1_img-invimg.pdf}
\caption{}
\end{figure}

\textbf{Remarks:}
\begin{enumerate}[i)]
\item For every $P\subseteq A$, $\emptyset \neq f_*(P)\subseteq B$ and $|P|\geqslant |f_*(P)|$. The \textbf{range} (or \textbf{image}) of $f$ is the set $f_*(A)$. The range need not be equal to the co-domain.
\item For every $Q\subseteq B$, $f^*(Q)\subseteq A$, possibly be empty, and $|Q|\leqslant |f^*(Q)|$. 
\item Given a function $f:A\rightarrow B$, the process of taking image (inverse image) of subsets of $A$ ($B$) can be thought of as operation of a new function $f_*:2^A\rightarrow 2^B$ ($f^*:2^B\rightarrow 2^A$) on subsets of $A$ ($B$) and induced by $f$.
\item Abuse of notation: 
\begin{enumerate}[a)]
\item Even though $f$ maps elements of $A$ to elements of $B$ and not subsets of $A$ (say $P$) to subsets of $B$ (say $Q$), often $f_*$ is replaced with $f$ for convenience, \textit{i.e.}, $f(P)$ is substituted for $f_*(P)$.
\item Similarly, $f^*$ is replaced with $f^{-1}$ for inverse image. Later, we will look at the \textbf{inverse} of a function $f:A\rightarrow B$ (if it exists) and denote it by $f^{-1}:B\rightarrow A$. If the inverse function of $f$ does not exist, then $f^{-1}(Q)$ is used to refer to the inverse image $f^*(Q)$ of $Q$ under $f$. If the inverse of $f$ exists, then it takes elements and not subsets of $B$ as the argument; and $f^*(Q)=f^{-1}(Q)=f^{-1}_*(Q)$, \textit{i.e.}, $f^{-1}(Q)$ can be used to refer to both the inverse image of $Q$ under $f$ ($f^*(Q)$) and the image of $Q$ under $f^{-1}$ ($f^{-1}_*(Q)$).
\end{enumerate}
\end{enumerate}

\begin{exmp}
Consider the function $f:\mathbb{R}\rightarrow \mathbb{R}$ plotted in Fig.~\ref{func_ex}.\begin{enumerate}[i)]
\item The range of $f$, $f_*(\mathbb{R})=[-3,\infty)\subset \mathbb{R}$.
\item For $P_1=[1.5,1.9]$ and $P_2=[-4.5,-3.3]$, $f_*(P_1)=[1.7,2.5]$ and $f_*(P_2)=[-3,-1]$.
\begin{figure}[h]
\centering
\includegraphics[scale=0.6]{Figures/l5f2_graph.pdf}
\caption{}
\label{func_ex}
\end{figure}
\item For $Q_1=[1.7,2.5]$ and $Q_2=[-4,-3.2]$, $f^*(Q_1)= [-2,-1.6] \cup [-0.6,0] \cup [1.7,2.5]$ and $f^*(Q_2)=\emptyset$.
\item From (i) and (ii), we have that $f^*(f_*(P_1))\neq P_1$, $f^*(f_*(P_2))=P_2$, $f_*(f^*(Q_1))=Q_1$ and $f_*(f^*(Q_2))=Q_2$ (cf. Theorem~\ref{img-preimg_props}).
\end{enumerate}
\end{exmp}

\begin{thm} Let $A$ and $B$ be sets, let $C,D\subseteq A$ and $S,T\subseteq B$ be subsets, and let $f:A\rightarrow B$ be a function. Let $I,J\neq\emptyset$, let $\{U_i:i\in I\}$ and $\{V_j:j\in J\}$ be indexed families of sets, where $U_i\subseteq A, \forall i\in I$ and $V_j\subseteq B, \forall j\in J$.
\begin{multicols}{2}
\begin{enumerate}[i)]
\item $f_*(\emptyset)=\emptyset$ and $f^*(\emptyset)=\emptyset$.
\item $f^*(B)=A$.
\item $f_*(C)\subseteq S$ iff $C\subseteq f^*(S)$.
\item If $C\subseteq C$ then $f_*(C)\subseteq f_*(D)$.
\item If $S\subseteq T$ then $f^*(S)\subseteq f^*(T)$.
\item $f_*(\bigcup_{i\in I}U_i)=\bigcup_{i\in I}f_*(U_i)$.
\item $f_*(\bigcap_{i\in I}U_i)\subseteq \bigcap_{i\in I}f_*(U_i)$.
\item $f^*(\bigcup_{j\in J}V_j)=\bigcup_{j\in J}f^*(V_j)$.
\item $f^*(\bigcap_{j\in J}V_j)=\bigcap_{j\in J}f^*(V_j)$.
\end{enumerate} 
\end{multicols}
\label{img-preimg_props}
\end{thm}
\noindent (\textit{Hint: Let $X$ and $Y$ be sets. To show $X=Y$, show that $X\subseteq Y$ and $Y\subseteq X$.})

\section{Composition}
\begin{defn}[Composition of Functions] Let $A$, $B$ and $C$ be sets, and let $f:A\rightarrow B$ and $g:B\rightarrow C$ be functions. The \textbf{composition} of $f$ and $g$ is the function $g\circ f:A\rightarrow C$ and is defined as follows:
\begin{equation*}
(g\circ f) (a)  =g(f(a))
\end{equation*}
for all $a\in A$
\end{defn}

\textbf{Remarks:}
\begin{enumerate}[i)]
\item Composition of three or more functions can be defined similarly.
\item Though read/written left to write, while obtaining value of $(g\circ f)(a),a\in A$, $f(a)$ is computed first followed by $g(f(a))$.
\item Function compositions can be visualized by \textit{commutative diagrams}. Following is the commutative diagram for $g\circ f$.
\begin{figure*}[h]
\centering
\includegraphics[scale=0.8]{Figures/l5f3_commdiag.pdf}
\end{figure*}
\item If the co-domain of the first function ($f$) is equal to the domain of the second ($g$), then the composition $g\circ f$ is defined. (The composition is defined iff the range of the first function is a subset of the domain of the second).
\item If $A=C$, then $f\circ g$ is also defined but need not necessarily be equal to $g\circ f$ or $1_A$.
\item The range of $g\circ f$ is a subset of range of $g$, \textit{i.e.}, $(g\circ f)_*\subseteq g_*$.
\end{enumerate}

\begin{exmp}
Let $f:\mathbb{R}\rightarrow \mathbb{R}$ be defined by $f(x)=x^3$ , $g:[0,\infty)\rightarrow \mathbb{R}$ be defined by $g(x)=\sqrt{x}$ and $h:\mathbb{R}\rightarrow \mathbb{R}$ be defined by $h(x)=2x$. Then 
\begin{enumerate}[i)]
\item $f\circ f:\mathbb{R}\rightarrow \mathbb{R}$ and $(f\circ f)(x)=(x^3)^3$.
\item $(f\circ g):[0,\infty)\rightarrow \mathbb{R}$ and $(f\circ g)(x)=\sqrt{x^3}$.
\item $(f\circ h):\mathbb{R}\rightarrow \mathbb{R}$ and $(f\circ h)(x)=(2x)^3$.
\item $(h\circ f):\mathbb{R}\rightarrow \mathbb{R}$ and $(h\circ f)(x)=2x^3$. (Note that $(h\circ f)\neq (f\circ h)$)
\item $(h\circ g):[0,\infty)\rightarrow \mathbb{R}$ and $(h\circ g)(x)=2\sqrt{x}$.
\item $f\circ h\circ g:[0,\infty)\rightarrow \mathbb{R}$ and $(f\circ h\circ g)(x)=(2\sqrt{x})^3$.
\item $f\circ f\circ h:\mathbb{R}\rightarrow \mathbb{R}$ and $(f\circ f\circ h)(x)=((2x)^3)^3$.
\item $(g\circ g),\,(g\circ f),\,(g\circ h),\,(f\circ g\circ h),\,(g\circ h\circ f),\,(g\circ f\circ h),\,(h\circ g\circ f)$ are not defined.
\end{enumerate}
Similarly, many more functions can be obtained.
\label{ex_comp}
\end{exmp}

\begin{defn}[Coordinate Function] Let $A,A_1,A_2,\ldots,A_n$ be sets for some $n\in \mathbb{N}$ and let $f:A\rightarrow A_1\times A_2\times \ldots \times A_n$ be a function. For each $i\in \{1,2,\ldots ,n\}$, let $f_i:A\rightarrow A_i$ be defined by $f_i=\Pi _i\circ f$, where $\Pi _i:A_1\times A_2\times \ldots \times A_n \rightarrow A$ is the $i^{th}$ projection map. Then, $f_1,f_2,\ldots ,f_n$ are the \textbf{coordinate functions} of $f$.
\end{defn}
Coordinates functions can be represented using a commutative diagram; given below is the commutative diagram for $n=2$.
\begin{figure*}[h]
\centering
\includegraphics[scale=0.56]{Figures/l5f4_coord.pdf}
\end{figure*}

\begin{exmp}
The function $f:\mathbb{R}^2\rightarrow \mathbb{R}^3$ defined by $f(x,y)=(xy,\,sin(x^2),\,x+y^3)$ has 3 coordinate functions $f_1,f_2,f_3:\mathbb{R}^2\rightarrow \mathbb{R}$ given by 
\begin{equation*}
f_1((x,y))=xy \qquad f_2((x,y))=sin(x^2) \qquad f_3((x,y))=x+y^3
\end{equation*}
\end{exmp}

\begin{lem}
Let $A,\,B,\,C,\,D$ be sets and $f:\rightarrow B,\, g:B\rightarrow C$ and $h:C\rightarrow D$ be functions. Then
\begin{enumerate}[i)]
\item $(h\circ g)\circ f=h\circ (g\circ f)$ \qquad (Associative law)
\item $f\circ 1_A=f$ and $1_B\circ f=f$ \qquad (Identity law)
\end{enumerate}
\end{lem}

\noindent Commutativity does not always hold for function composition (cf. Ex.~\ref{ex_comp}(iv)).

\section{Inverse Function}
\begin{defn}
Let $A$ and $B$ be sets and let $f:A\rightarrow  B$ and $g:B\rightarrow  A$ be functions. Then the function $g$ is 
\begin{enumerate}[i)]
\item a \textbf{right inverse} for $f$ if $f\circ g=1_B$,
\item a \textbf{left inverse} for $f$ if $g\circ f=1_A$ and
\item an \textbf{inverse} for $f$ if it is both a right and left inverse.
\end{enumerate}
\end{defn}
\noindent If $g$ is a left (right) inverse for $f$, then $f$ is a right (left) inverse for $g$.

\begin{exmp}
Let $P$ be the set of all people and $W$ be the set of women with at least one child, and let $c:P\rightarrow W$ be the function that maps a person to their mother and $m:W\rightarrow P$ be the function that maps a woman to her eldest child. 

Choose a person $p\in P$ who is not the eldest of their siblings, and let the eldest sibling of the chosen person be $p'$. Then, $c(p)=w$, for some $w\in W$ and $m(w)=p'\neq p$. Thus, $(m\circ c)(p)\neq p, \forall p\in P$.

Choose a woman $w\in W$. Then, $m(w)=p$ for some $p\in P$ and $c(p)=w$. Thus, $(c\circ m)(w)=w$.

Hence, $m$ has a left inverse ($c$) but no right inverse and $c$ has a right inverse ($m$) but no left inverse.

Had $P$ been the set of people who are eldest of their siblings, then the maps $c$ and $m$ would have been inverse of each other.

Had $m$ been a map from $P$ to $P$, then neither right nor left inverse of either maps would not have existed.
\end{exmp}

\begin{lem}
Let $A$ and $B$ be sets, and let $f:A\rightarrow B$ be a function.
\begin{enumerate}[i)]
\item If $f$ has an inverse, then it is unique.
\item If $f$ has a right inverse $g$ and a left inverse $h$, then $g=h$, and hence $f$ has an inverse.
\item If $f$ has an inverse $g$, then $g$ has an inverse, which is $f$.
\end{enumerate}
\end{lem}

\begin{defn}
Let $A$ and $B$ be sets, and let $f:A\rightarrow B$ be a function. If $f$ has an inverse, then the inverse is denoted by $f^{-1}:B\rightarrow A$.
\end{defn}

\noindent \textbf{Remarks:} 
\begin{enumerate}
\item $f^{-1}\circ f=1_A$ and $(f^{-1}\circ f)(a)=a,\,\forall a\in A$.
\item $f\circ f^{-1}=1_B$ and $(f\circ f^{-1})(b)=b,\,\forall b\in B$.
\item Given the graph of a function $f:A\rightarrow B$, where $A,B\subseteq \mathbb{R}$, the inverse function $f^{-1}$ can be plotted by reflecting the graph of $f$ in the line $x=y$. This is same as first reflecting $f$ in \textit{y-axis} followed by $90^{\circ}$ clockwise rotation.
\end{enumerate}


\end{document}
