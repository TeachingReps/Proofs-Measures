\documentclass[a4paper,english,12pt]{article}   	% use "amsart" instead of "article" for AMSLaTeX format
\usepackage{%
	amsmath,%
	amsfonts,%
	amssymb,%
	amsthm,%
	hyperref,%
	url,%
	latexsym,%
	epsfig,%
	graphicx,%
	psfrag,%
	subfigure,%	
	color,%
	tikz,%
	pgf,%
	pgfplots,%
	pgfplotstable,%
	pgfpages,%
	proofs%
}

\usepgflibrary{shapes}
\usetikzlibrary{%
  arrows,%
	backgrounds,%
	chains,%
	decorations.pathmorphing,% /pgf/decoration/random steps | erste Graphik
	decorations.text,%
	matrix,%
  positioning,% wg. " of "
  fit,%
	patterns,%
  petri,%
	plotmarks,%
  scopes,%
	shadows,%
  shapes.misc,% wg. rounded rectangle
  shapes.arrows,%
	shapes.callouts,%
  shapes%
}

\theoremstyle{plain}
\newtheorem{thm}{Theorem}[section]
\newtheorem{lem}[thm]{Lemma}
\newtheorem{prop}[thm]{Proposition}
\newtheorem{cor}[thm]{Corollary}

\theoremstyle{definition}
\newtheorem{defn}[thm]{Definition}
\newtheorem{conj}[thm]{Conjecture}
\newtheorem{exmp}[thm]{Example}
\newtheorem{assum}[thm]{Assumptions}

%\theoremstyle{remark}
\newtheorem{rem}{Remark}
\newtheorem{note}{Note}

\makeatletter
\def\th@plain{%
  \thm@notefont{}% same as heading font
  \itshape % body font
}
\def\th@definition{%
  \thm@notefont{}% same as heading font
  \normalfont % body font
}
\makeatother
\date{}

\title{Lecture 18 :  Product topology}
\author{}

\begin{document}
\maketitle
We have seen topology on $X \times Y$ where $X,Y$ are two topological spaces. Next  step is to generalize this to the products of the form $X_{1} \times \ldots  \times X_{n}$ and $X_{1} \times X_{2} \times \ldots$. One way is to consider the basis of the forms $U_{1} \times \ldots  \times U_{n}$ and $U_{1} \times U_{2} \times \ldots$ where $U_{i}$ are open sets in $X_{i}$. This procedure defines topology named Box topology. Another way is to generalize through the sub basis definition which results in topology called Product topology. Let us define these formally as  
\section{Box and Product topology} 
Let $I$ be an index set, and $\{X_{i} \colon i \in I\}$ an index family of sets.
\begin{defn}[Box topology]
Let $(X_{i},\mathcal{T}_{i})$ be the topological space for each $i \in I$. Let $\mathcal{B} = \{\prod_{i \in I} U_{i} \colon U_{i} \in \mathcal{T}_{i} , \forall i \in I \}$ be a basis, then the defined topology is called \textbf{Box topology}.
\end{defn}
To check $\mathcal{B}$ indeed a basis
\begin{enumerate}
\item $X = \prod_{i \in I} X_{i} $ is itself a basis element, hence $X$ is covered by $\mathcal{B}$.
\item Let $\prod_{i \in I} U_{i}$ and $\prod_{i \in I} V_{i}$ be the two elements of basis $\mathcal{B}$ then $\prod_{i \in I} U_{i} \cap \prod_{i \in I} V_{i} = \prod_{i \in I} (U_{i} \cap V_{i}) \in \mathcal{B}$.
\end{enumerate}

\begin{defn}[Product topology]
Let $(X_{i},\mathcal{T}_{i})$ be the topological space for each $i \in I$. Let $\mathcal{S} = \cup_{i \in I} \{\pi_{i}^{-1} (U_{i})  \colon U_{i} \in \mathcal{T}_{i}  \}$ be a sub basis, then the defined topology is called \textbf{Product topology}.
\end{defn}

\begin{thm}[Comparision of box and product topologies]
The basis for the  box topology is of the form 
\begin{align*}
\mathcal{B}_{b} = \{\prod_{i \in I} U_{i} \colon U_{i} \in \mathcal{T}_{i} , \forall i \in I \}.
\end{align*}
The basis for the  product topology is of the form 
\begin{align*}
\mathcal{B}_{p} &= \{ \cap_{k=1}^{n} \pi_{i_{k}}^{-1}(U_{i_{k}}) \colon i_{k} \in I ~and ~n ~finite  \} \\ &= \{ \prod_{i \in I} U_{i} \colon U_{i} = X_{i}  ~\forall i \in I \setminus \{i_{1},\ldots,i_{n}\} ~for ~some ~n\}.
\end{align*}
\end{thm}
\begin{lem}
$\mathcal{T}_{p} = \mathcal{T}_{b}$ for $\prod_{i =1}^{n} X_{i}$ and $\mathcal{T}_{p} = \mathcal{T}_{b}$ in general.
\end{lem}


\begin{thm}
Let $\mathcal{B}_{i}$ be basis for topological space $(X_{i} , \mathcal{T}_{i})$ then $\mathcal{B}_{b} = \{\prod_{i \in I} B_{i} \colon B_{i} \in \mathcal{B}_{i} ~\forall i \in I\}$ is a basis for box topology.  $\mathcal{B}_{p} = \{\prod_{i \in I} B_{i} \colon B_{i} \in \mathcal{B}_{i}, i \in \{i_{1},\ldots,i_{n}\}  \\ ~for ~some ~finite ~subset ~of ~I ~and ~ B_{i} = X_{i}, ~ i \in I\setminus \{i_{1},\ldots,i_{n}\}  \}$ is a basis for product topology.
\end{thm}
\begin{proof}
To prove the first part, Let $U \subseteq X$ be open. Then by construction of box topology for each $x \in U$ we can find $\prod_{i \in I} U_{i}$, $U_{i}$ open in $X_{i}$, and $x \in \prod_{i \in I} U_{i} \subseteq U$. Then by basis definition we can find $x_{i} \in B_{i} \in \mathcal{B}_{i} $ such that $ B_{i} \subseteq U_{i}$. Which completes the proof since $x \in \prod_{i \in I} B_{i} \subseteq \prod_{i \in I} U_{i} \subseteq U$ and $\prod_{i \in I} B_{i} \in \mathcal{B}_{b}$ is indeed a basis for box topology.\\
 Next, consider  $U \subseteq X$ be open in product topology. Then for each $x \in U$ we can find $\prod_{i \in I} U_{i}$, $U_{i}$ open in $X_{i}$ for each $i \in I$ and for all but finitely many $U_{i} = X_{i}$. Call them $i_{1},i_{2},\ldots,i_{n}$. We can find $B_{i_{k}} \in \mathcal{B}_{i_{k}}$ such that $x_{i_{k}} \in B_{i_{k}} \subseteq U_{i_{k}}$ for $k = 1,2,\ldots,n$. Then $x \in \prod_{i \in I} V_{i} \subseteq \prod_{i \in I} U_{i} \subseteq U$, where 
 \begin{align*}
V_{i} = \begin{cases} B_{i}, &i \in \{ i_{1},i_{2},\ldots,i_{n}\}, \\ X_{i}, & i \in I \setminus \{ i_{1},i_{2},\ldots,i_{n}\}. \end{cases}
\end{align*}
Which completes the proof since $\prod_{i \in I} V_{i} \in \mathcal{B}_{p}$ which is a basis for product topology.
 \end{proof}
\begin{exmp}
Consider space $\mathbb{R}^{n}$.  We know that basis for $\mathbb{R}$ consists of all open intervals. Also, basis for topology $\mathbb{R}^{n}$ is of the form $\mathcal{B} = \{(a_{1} , b_{1}) \times (a_{2} , b_{2}) \times \ldots \times (a_{n} , b_{n}) \colon a_{i},b_{i} \in \mathbb{R}, \forall i \in [n]\}$
\end{exmp}
\begin{rem} Box and Product topologies agree on $\mathbb{R}^{n}$.
\end{rem}
\begin{thm}
Let $A_{i}$ be subspace of $X_{i},~ \forall i \in I$ then $A = \prod_{i \in I} A_{i}$ has a subspace topology of $X = \prod_{i \in I} X_{i}$ under box or product topologies. 
\end{thm}
\begin{proof}
First consider the case of product topology. Let $\mathcal{T}_{s},\mathcal{T}_{p}$ be the topologies for $A$ inherits as subspace of $X$ and as a product space respectively.
Then basis for $\mathcal{T}_{s},\mathcal{T}_{p}$ are $\mathcal{B}_{s} = \{ B \cap A \colon B \in \mathcal{B}\}$, where $\mathcal{B}$ is basis for product topology on X and $\mathcal{B}_{p} = \{\prod_{i \in I} O_{i} \colon O_{i} ~\text {open in } ~A_{i}, \text { and equals} ~A_{i} ~\text {for all but finitely many} ~i \}$. Need to show both generates same topology. So consider $x \in B \in \mathcal{B}_{s}$ then $B = A \cap \prod_{i \in I} V_{i} = \prod_{i \in I} (V_{i} \cap A_{i})$, where $V_{i}$ is open in $X_{i}$ and $V_{i} \cap A_{i}$ is open in $A_{i}$. Since $V_{i} = X_{i}$ for all but finitely many $i$. So $V_{i} \cap A_{i} = A_{i}$ for those $i$. So $B \in \mathcal{B}_{p}$. To prove the other way consider $B \in \mathcal{B}_{p}$. Then $B = \prod_{i \in I} O_{i}$ where $O_{i}$ is open in $A_{i}$. Which can be written as $O_{i} = V_{i} \cap A_{i}$ where $V_{i}$ is open in $X_{i}$. Then $B = \prod_{i \in I} O_{i} = \prod_{i \in I} (V_{i} \cap A_{i}) = \prod_{i \in I} V_{i} \cap A \in \mathcal{B}_{s}$.\\ The proof for box topology also falls  in the similar lines.
\end{proof}
\begin{thm}
If $X_{i}$ is Hausdorff $~\forall i \in I$, then $X = \prod_{i \in I} X_{i}$ is Hausdorff both in product and box topologies.
\end{thm}
\begin{proof}
The proof is similar  for both box and product topology. Let $x,y \in X$ be distinct then $x_{i} \neq y_{i}$ for some $i \in I$. Since $X_{i}$ is Hausdorff we can find two disjoint open neighborhoods $U,V$ for $x_{i},y_{i}$ respectively. Then $\pi_{i}^{-1}(U),\pi_{i}^{-1}(V)$ are disjoint open neighborhoods of $x,y$ respectively. Hence $X$ is Hausdorff.
\end{proof}
\begin{thm}
Let $A_{i} \subseteq X_{i} ~\forall i \in I$. If $\prod_{i \in I} X_{i}$ is given either product or box topology then $\prod_{i \in I} \bar{A_{i}} = \overline{\prod_{i \in I} A_{i}}$.
\end{thm}
\begin{proof}
$(\Rightarrow)$ Let $x \in \prod_{i \in I} \bar{A_{i}}$, we can choose $y_{i} \in U_{i} \cap A_{i}$ for each $i \in I$, where $U_{i}$ is neighborhood of $x_{i}$. Then $y \in U \cap \prod_{i \in I} A_{i}$, where $U$ is neighborhood of $x$. Therefore $x \in \overline{\prod_{i \in I} A_{i}}$.\\
$(\Leftarrow)$ Conversely, let $x \in \overline{\prod_{i \in I} A_{i}}$ and let $V_{i}$ be a neighborhood of $x_{i}$  in $X_{i}$. Since, $\pi_{i}^{-1} (V_{i})$ is open neighborhood of $x$ it contains a point $y \in \prod_{i \in I} A_{i}$. Then $\pi_{i} (y) = y_{i} \in V_{i} \cap A_{i}$. It follows that $x_{i} \in \bar{A}_{i}$ 
\end{proof}
\begin{lem}
$\pi_{i}$ is continuous in product topology.
\end{lem}
\begin{proof}
$\pi_{i}^{-1} (U_{i}) \in \mathcal{S}$ for all $U_{i}$ open in $X_{i}$. Since an element of $\mathcal{S}$ is open, $\pi_{i}$ is continuous.
\end{proof}
\begin{thm}
Let $f \colon A \to \prod_{i \in I} X_{i}$ given by $f(a) = \{ f_{i}(a) \colon i \in I \}$ where $f_{i} \colon A \to X_{i} ~ \forall i \in I$. Let $\prod_{i \in I} X_{i}$ have the product topology then $f$ is continuous if and only if $f_{i}$ is continuous for all $i \in I$. 
\end{thm}
\begin{proof}
$(\Rightarrow)$ Since $f$ is continuous we know that $f_{i} = \pi_{i} \circ f$. From previous lemma we know that $\pi_{i}$ is continuous. Hence $f_{i}$ is continuous because composition of two continuous functions is continuous.\\
$(\Leftarrow)$ Let $f_{i}$ is continuous for each $i \in I$. To prove $f$ is continuous it suffices to show that inverse image of $f$ under each sub basis element is open in $A$. We know that element of the sub basis for product topology on $\prod_{i \in I} X_{i}$ is a set of the form $\pi_{i}^{-1} (U_{i})$ for some $i \in I$.  Let $s \in \mathcal{S}$, then $f^{-1}(s) = f^{-1}(\pi_{i}^{-1}(U_{i})) = f_{i}^{-1}(U_{i})$, which is open in $A$ since $f_{i}$ is continuous. 
\end{proof} 
The above theorem does not follow for box topology. To see it consider   the following example  
\begin{exmp}
Consider $\mathbb{R}^{\mathbb{N}} = \prod_{n \in \mathbb{N}} X_{n}$ where $X_{n} = \mathbb{R}$. Let $f \colon \mathbb{R} \to \mathbb{R}^{\mathbb{N}}$ is defined as $f(t) = (t,t,t,\ldots)$ and $f_{n}(t) = t$. Then, $f_{n}$ is continuous for each $n \in \mathbb{N}$. Hence $f$ is conyinuous in product topology. We will show that $f$ is not continuous in box topology. \par
Consider, $B = \prod_{n \in \mathbb{N}} (-\frac{1}{n} , \frac{1}{n})$ open in $X$ for box topology.\\
\textbf{Claim:} $f^{-1}(B)$ is not open.
\begin{proof}
If $f^{-1}(B)$ is open in then $\exists \delta > 0$ such that $(-\delta,\delta) \subseteq f^{-1}(B)$. That is $f(-\delta,\delta) \subseteq B$ or $f_{n}(-\delta,\delta) \subseteq B_{n} ~\forall n \in \mathbb{N}$. Which is a contradiction.   
\end{proof}
\end{exmp}

\section{Metric topology}
\begin{defn} 
A \textbf{metric} on a set $X$ is function, $d \colon X \times X \to \mathbb{R}$ with the following properties: 
\begin{enumerate}
\item(Positivity) $d(x,y) \geq 0 ~\forall x,y \in X$. Equality holds if and only if $x= y$.
\item(Symmetry) $d(x,y) = d(y,x) ~\forall x,y \in X$.
\item(Triangle inequality) $d(x,y) \leq d(x,z) + d(z,y) ~\forall x,y,z \in X$
\end{enumerate}
\end{defn}
\par Given a metric $d$ on $X$, the number $d(x,y)$ is often called the \textbf{distance} between $x$ and $y$ in the metric $'d'$. Given $\epsilon > 0$, consider the set $B_{d}(x,\epsilon) = \{y \in X \colon d(x,y) < \epsilon\}$. Of all points $y$ whose distance from $X$ is less than $\epsilon$. It is called the \textbf{$\epsilon$ - ball centered at X}.
\begin{lem}
If $y \in B(x,\epsilon)$ then $\exists \delta > 0$ such that $B(y,\delta) \subseteq B(x,\epsilon)$.
\end{lem}
\begin{proof}
Define $\delta = \epsilon - d(x,y)$ be positive number. consider $z \in B(y,\delta)$, then $d(y,z) < \epsilon - d(x,y)$. So we can conclude that $d(x,z) \leq d(x,y) + d(y,z) < \epsilon$. Which imply that $z \in B(X,\epsilon)$. Then $B(y,\delta) \subset B(x,\epsilon)$. 
\end{proof}
\begin{defn}
If $d$ is a metric on $X$, $\mathcal{B} = \{B_{d}(x,\epsilon) \colon x \in X , \epsilon > 0\}$ basis for topology on $X$, the defined topology is called the \textbf{metric topology} induced by $d$.
\end{defn}
To check $\mathcal{B}$ indeed a basis
\begin{enumerate}
\item $x \in B(x,\epsilon)$ for any $\epsilon > 0$.
\item Let $B_{1},B_{2} \in \mathcal{B}$ and $y \in B_{1} \cap B_{2}$. Choose $\delta_{1},\delta_{2} > 0$ such that $B(y,\delta_{1}) \subseteq B_{1}$, $B(y,\delta_{2}) \subseteq B_{2}$. Let $\delta = \delta_{1} \wedge \delta_{2}$, then $B(y,\delta) \subseteq B_{1} \cap B_{2}$. 
\end{enumerate}



%\section{}
%\subsection{}



\end{document}  