\documentclass[a4paper,english,12pt]{article}
\usepackage{%
	amsfonts,%
	amsmath,%	
	amssymb,%
	amsthm,%
	bbm,%
	biblatex,%
	caption,%
	color,%
	enumerate,%
	epsfig,%
	epstopdf,%
	geometry,%
	graphicx,%
	hyperref,%
	latexsym,%
	mathtools,%
	multicol,%
	pgf,%
	%pgfplots%
	%pgfplotstable,%
	pgfpages,%
	proof,%
	psfrag,%
	subfigure,%	
	tikz,%
	ulem,%
	url%
}

\usepackage[mathscr]{eucal}
\usepgflibrary{shapes}
\usetikzlibrary{%
  arrows,%
	backgrounds,%
	chains,%
	decorations.pathmorphing,% /pgf/decoration/random steps | erste Graphik
	decorations.text,%
	matrix,%
  positioning,% wg. " of "
  fit,%
	patterns,%
  petri,%
	plotmarks,%
  scopes,%
	shadows,%
  shapes.misc,% wg. rounded rectangle
  shapes.arrows,%
	shapes.callouts,%
  shapes%
}

\theoremstyle{plain}
\newtheorem{thm}{Theorem}[section]
\newtheorem{lem}[thm]{Lemma}
\newtheorem{prop}[thm]{Proposition}
\newtheorem{cor}[thm]{Corollary}

\theoremstyle{definition}
\newtheorem{defn}[thm]{Definition}
\newtheorem{conj}[thm]{Conjecture}
\newtheorem{exmp}[thm]{Example}
\newtheorem{assum}[thm]{Assumptions}

\theoremstyle{remark}
\newtheorem{rem}{Remark}
\newtheorem{note}{Note}

\newcommand{\norm}[1]{\left\lVert#1\right\rVert}
\newcommand{\tr}{\operatorname{tr}}
\newcommand{\Real}{\mathbb{R}}

\makeatletter
\def\th@plain{%
  \thm@notefont{}% same as heading font
  \itshape % body font
}
\def\th@definition{%
  \thm@notefont{}% same as heading font
  \normalfont % body font
}
\makeatother
\date{}

\title{Lecture 7: Relations}
\author{}

\begin{document}
\maketitle

\section{Relation}
Relation between two objects signify some connection between them. For example, relation of one person being biological parent of another. If we take any two people at random, say persons $X$ and $Y$, then either $X$ is a parent of $Y$ or not. This depends on our knowledge of knowing whether $(X,Y)$ is a parent child pair. Alternatively, we can list all pairs of people $(X,Y)$ such that $X$ is a parent of $Y$. Then, knowing whether $X$ and $Y$ are related is equivalent to checking whether pair $(X,Y)$  belongs to this set of listed pairs. We formalize this notion below.
\begin{defn}[Relation] Let $A$ and $B$ be sets. A \textbf{relation} $R$ from $A$ to $B$ is a subset
\begin{equation*}
R \subseteq A \times B. 
\end{equation*}
If $a \in A$ and $b \in B$, we write $aRb$ if $(a,b) \in R$ and $a \centernot R b$ if $(a,b) \notin R$. A relation from $A$ to $A$ is called a relation on $A$.
\end{defn}
\begin{exmp} Following are some examples of relations.
\begin{enumerate}
\item Let $A=\{1,2,3\}$ and $B =\{x,y,z\}$. Then, we can define a relation $S = \{(1,y),(1,z),(2,y)\}$ such that $1Sy, 1Sz, 2Sy$.
\item Let $P$ be set of all people. Then, we define a relation $R$ as $R \subseteq P \times P$ such that $R =\{(x,y) \in P \times P :x \text{ and } y \text{ have at least one parent in common}\}$.
\item Symbols $<$ and $\leq$ represent relations on $\mathbb{R}$.
\item Let P be set of all people, and B be set of all books. Then, we can define a relation $T \subseteq P \times B$ such that $T=\{(x,y) \in P \times B: x \text{ has read } y \}$. 
\item Let $A$ be a set. Define a relation on $\mathcal{P}(A)$, by saying that $P,Q \in \mathcal{P}(A)$ are related iff $P \subseteq Q$. Let $U =\{(P,Q) \in \mathcal{P}(A) \times \mathcal{P}(A) : P \subseteq Q\}$.
\item A function $f: A \to B$ is subset of $A \times B$ satisfying certain conditions. Hence, a function $f: A \to B$ is also a relation from $A$ to $B$.
\end{enumerate}
\end{exmp}
\begin{rem} Relations are more general than functions. Notice the difference between functions and relations. The following two properties that can hold for relations but not for functions.
\begin{enumerate}
	\item A relation can have no ordered pair for some $a \in A$.
	\item A relation can have more than one ordered pair for some $a \in A$.
\end{enumerate}
\end{rem}
\begin{defn}[Relation Class] Let $A$ and $B$ be non-empty sets and let $R$ be a relation from $A$ to $B$.For each element $x \in A$, define the \textbf{relation class} of $x$ with respect to relation $R$, denoted $R[x]$, to be the set
\begin{equation*} 
R[x] =\{y \in B : x R y \}.
\end{equation*} 
If the relation $R$ is understood from the context, we will often write $[x]$ instead of $R[x]$.
\end{defn}
\begin{exmp} Let's consider the previous examples.
\begin{enumerate}
	\item For this relation we see that $[1]=\{y,z\}$, $[2]=\{y\}$, and $[3] = \emptyset$. We have an example of relation class that is empty.
	\item There are a number of distinct cases here, and we will examine a few of them. 
	\begin{enumerate}
		\item If $x$ is the only child of each of her parents, then $[x] = \{x\}$, where we observe that
$x$ has the same parents as herself. Notice that for any two distinct single children, $[x] \cap [y] = \emptyset$. That is, we have examples of relations classes that are disjoint.
		\item If $y$ and $z$ are the only two children of each of their parents, then $[y] = \{y, z\} = [z]$. 
		\item If $a$ has one half-sibling $b$ by her father, and another half-sibling $c$ by her mother, and each of $b$ and $c$ have no other siblings or half-siblings, then $[a] = \{a, b, c\}$, and $[b] = \{a, b\}$, and $[c] = \{a, c\}$.
	\end{enumerate}
	\item For the relation $<$, we see that $[x] = (x, \infty)$ for all $x \in \mathbb{R}$, and for the relation
$\leq$, we see that $[x] = [x, \infty)$ for all $x \in \mathbb{R}$.
	\item Relation class of a person $x$ is $[x] = \{\text{ set of books read by }x\}$. 
	\item Relation class of a subset $P \subseteq A$ is $[P] = \{ Q \subset A: P \subseteq Q \}$.
	\item Relation class of a function $x$ is $[x] = \{f(x)\}$.
\end{enumerate}
\end{exmp}
We see that relation classes can be empty. Two relation classes can be disjoint or have non-empty intersection. This is due to the fact that relations are very general and have no structures. In the following we would add some structure to define two specific relations.
\begin{defn} Let $A \neq \emptyset$ and let $R$ be a relation on $A$.
	\begin{enumerate}
		\item Relation $R$ is \textbf{reflexive} if $xRx$ for all $x \in A$.
		\item Relation $R$ is \textbf{non-reflexive} if $xRx$ doesn't hold for any $x \in A$.
		\item Relation $R$ is \textbf{symmetric} if $xRy$ implies $yRx$ for all $x \in A$.
		\item Relation $R$ is \textbf{transitive} if $xRy$ and $yRz$ implies $xRz$ for all $x,y,z \in A$.
		\item Relation $R$ is \textbf{comparable} if $x \neq y$, then either $xRy$ or $yRx$ for all $x,y \in A$.
	\end{enumerate}
\end{defn}
\begin{exmp} We look at some relations and which of the above-defined properties they satisfy. 
\begin{enumerate}
	\item Congruence of triangles is reflexive, symmetric, and transitive.
	\item Relation of one person weighing within $5$ kgs of another person is reflexive, symmetric, but not transitive.
	\item The relation $\leq$ on real numbers is not symmetric. However, this relation is reflexive, and transitive. 
	\item Let $C=\{1,2,3\}$, and a relation $P$ on set $C$ such that 
	\begin{align*}
	P=\{(2,2),(3,3),(2,3),(3,2)\}.
	\end{align*}
	Then $P$ is not reflexive as $(1,1) \notin P$, symmetric as $\{(2,3),(3,2)\} \in P$, and transitive as $\{(2,3),(3,2),(2,2)\} \in P$.
	\item Let $B=\{x,y,z\}$ and a relation $T$ on set $B$ such that 
	\begin{align*}
	T = \{(x,x),(y,y),(z,z),(x,y),(y,z)\}.
	\end{align*}
	Relation $T$ is reflexive as $\{(x,x),(y,y),(z,z)\} \in T$, not symmetric as $\{(y,x),(z,y)\} \notin T$, and not transitive as $(x,z) \notin T$.
	\item Relation of one person being cousin of another is symmetric, but not reflexive, or transitive. In fact, it is non-reflexive.
	\item Relation $<$ on real numbers is neither reflexive, nor symmetric. However, it is non-reflexive, comparable, and transitive.
	\item Relation of one person being daughter of another person is neither reflexive, nor symmetric, nor transitive.
\end{enumerate}
\end{exmp}

\section{Order Relations}
First interesting relation we will study is order relation.
\begin{defn}[Order Relation] A relation $C$ on a set $A$ is called an \textbf{order relation} (or a \textbf{simple order}, or a \textbf{linear order}) if it is comparable, non-reflexive, and transitive. Symbol $<$ is commonly used to denote an order relation.
\end{defn}
\begin{rem} We use following facts and conventions for order relations.
\begin{enumerate}
	\item Comparability doesn't prohibit $xCy$ and $yCx$ both to be true. However, transitivity and non-reflexivity of the order relation prohibits such an occurrence.
	\item Statement $x \leq y$ stands for statement ``either $x < y$ or $x = y$''.
	\item Statement $x < y < z$ stands for $x < y$ and $y < z$.
\end{enumerate}
\end{rem}
\begin{exmp} Following are some of the examples for order relations.
	\begin{enumerate}
		\item Let $R = \{(x,y) \in \mathbb{R}^2: x < y\}$. This is an order relation, called \textbf{usual order relation} on $\mathbb{R}$.
		\item Let $C = \{(x,y) \in \mathbb{R}^2: x^2 < y^2, \text{ or } x = y \text{ and } x < y\}$. This can be verified to be an order relation.
		\item Let $P$ be set of all people in the world. We respectively define descendent, blood, and sibling relations as 
		\begin{align*}
		D &= \{(x,y) \in P \times P : x \text{ descendent of } y\},\\
		B &= \{(x,y) \in P \times P : x \text{ has an ancestor, who is also an ancestor of } y\},\\
		S &= \{(x,y) \in P \times P : \text{ parents of } x \text{ are parents of } y\}.
		\end{align*}
		Descendent relation is not comparable, but it is non-reflexive, and transitive. Blood relation is neither comparable, nor non-reflexive, nor transitive. Sibling relation is transitive, but neither non-reflexive, nor comparable.  
		\item Let $A$ be a non-empty set. We can define a relation on power set of $A$ as $\{(S,T) \in \mathcal{P}(A)\times \mathcal{P}(A): S \subset T\}$. This relation is not comparable, but non-reflexive and transitive.
	\end{enumerate}
\end{exmp}
\begin{defn}[Strict Partial Order] Relations that satisfy non-reflexivity and  transitivity are called \textbf{strict partial order} relations.
\end{defn}
\begin{defn} Let $X$ be a set, and $<$ an order relation on $X$, with $a < b$ elements of $X$. Then, an \textbf{open interval} in $X$ is defined as set of elements $\{x \in X : a < x < b\}$, denoted by $(a, b)$. If this set is empty, we call $a$ the \textbf{immediate predecessor} of $b$, and we call $b$ the \textbf{immediate successor} of $a$.
\end{defn}

\section{Order Relation}  

Before introducing order topology, we first define an order relation.
\begin{defn}[Order relation]
A relation $C$ on set $A$ is called an \textbf{order relation} (or  \textbf{simple order} or \textbf{linear} order) if it satisfies the following properties:
\begin{enumerate}[i)]
\item (Comparability) For every $x, y \in A$ for which $x \neq y$, either $x C$ or $y C x$.
\item (Non-reflexivity) For no $x \in A$ does the relation $x C x$ holds.
\item (Transitivity) If $x C y$ and $y C z$, then $x C z$.
\end{enumerate}
\end{defn}

\begin{exmp}
Consider the relation on the real line consisting of all pairs $(x, y)$ of the real numbers such that $x < y$. It is an order relation, called the \textbf{usual order relation} on the real line.
\end{exmp}


\begin{rem}
\begin{enumerate}
\item A relation is called a \textbf{strict partial order} if it satisfies only the non-reflexivity and the transitivity properties of order relation.    
\item The symbol $<$ is commonly used to denote an order relation.
\item $x \le y$ implies that either $x = y$ or $x < y$. 
\end{enumerate}
\end{rem}

\begin{defn}
If $X$ is a set, and $<$ is an order relation on $X$, and if $a < b$ for $a, b \in X$. We use the notation $(a,b)$ to denote the set $ \{ x \in X: a < x < b \}$; it is an open interval in $X$. If this set is empty, we call $a$, the \textbf{immediate predecessor} of $b$, and we call $b$, the \textbf{immediate successor} of $a$.
\end{defn}


\begin{defn}
Let $A$ and $B$ be sets with order relations $<_{A}$ and $<_{B}$, respectively. We say that sets $A$ and $B$ have the same \textbf{order type} if there is a bijective correspondence between them that preserves order. That is, there is a bijective function $f: A \to B$ such that $a_{1} <_{A} a_{2}$ implies $f(a_{1}) <_{B} f(a_{2})$.
\end{defn}

\begin{exmp}
\begin{enumerate}[a)]
\item $(-1, 1)$ and $\R$ have the same order type as there exists a bijective map $f:(-1, 1) \to \R$, $f(x) = \frac{x}{1 - x^{2}}$ which preserves order.
\item $A = \{0\} \cup (1, 2)$ has the same order type as $[0, 1)$ as there exists a bijective map $f: A \to [0, 1)$ which preserves order. The function $f$ is such that $f(0) = 0$ and $f(x) = x-1$ for $x \in (1, 2)$.

\end{enumerate}
\end{exmp}


\begin{defn} [Dictionary order]
Suppose $A, B$ are two sets with order relation $<_{A}$ and $<_{B}$, respectively. Then, we can define an order relation on $A \times B$ by defining: 
\begin{equation}
a_{1} \times b_{1} < a_{2} \times b_{2} \nonumber
\end{equation}
if $a_{1} <_{A} a_{2}$, or if $a_{1} = a_{2}$ and $b_{1} <_{B} b_{2}$. It is called the \textbf{dictionary order relation} on the set $A \times B$. 
\end{defn}

One of the properties of the real numbers is the \textbf{least upper bound property}. We will now define a similar property for an arbitrary but ordered set. We start by defining the largest and smallest elements of an arbitrary ordered set. Let $A$ be a set ordered by the relation $<$. Let $A_{o}$ be a subset of $A$.  We say that the element $b$ is the \textbf{largest element} of $A_{o}$ if $b \in A_{o}$ and if $x \le b$ for every $x \in A_{o}$. Likewise, we say that $a$ is the \textbf{smallest element} of $A_{o}$ if $a \in A_{o}$ and if $a \le x$ for every $x \in A_{o}$. A set has at most one largest and one smallest element. In many cases, a smallest/largest element may not exist, for example, the open interval $(0,1)$. 

\begin{defn}
The set $A_{o} \subseteq A$ is \textbf{bounded above} if there exists and element $b \in A$ such that, $x \le b$ for any element $x \in A_{o}$. This element is called an \textbf{upper bound} for $A_{o}$. If the set of all upper bounds on $A_{o}$ has a smallest element $b_{o}$, then $b_{o}$ is called the \textbf{least upper bound} or \textbf{supremum} of $A_{o}$, denoted by $sup A_{o}$.  

If $\sup A_{o} \in A_{o}$, then it is the largest element in $A_{o}$.
\end{defn}

\begin{defn}
An ordered set is said to have the \textbf{least upper bound property} if every non empty subset $A_{o}$ of $A$ that is bounded above has a least upper bound. Likewise, the set $A$ is said to have the \textbf{greatest lower bound property} if every nonempty subset $A_{o}$ of $A$ that is bounded below has a greatest lower bound. 
\end{defn}

\begin{rem} A set has least upper bound property if and only if it has the greatest lower bound property. 
\end{rem}(Show proof for this claim)

\end{document}
