\documentclass[a4paper,english,12pt]{article}
\usepackage{%
	amsmath,%
	amsfonts,%
	amssymb,%
	amsthm,%
	hyperref,%
	url,%
	latexsym,%
	epsfig,%
	graphicx,%
	psfrag,%
	subfigure,%	
	color,%
	tikz,%
	pgf,%
	pgfplots,%
	pgfplotstable,%
	pgfpages,%
	proofs%
}

\usepgflibrary{shapes}
\usetikzlibrary{%
  arrows,%
	backgrounds,%
	chains,%
	decorations.pathmorphing,% /pgf/decoration/random steps | erste Graphik
	decorations.text,%
	matrix,%
  positioning,% wg. " of "
  fit,%
	patterns,%
  petri,%
	plotmarks,%
  scopes,%
	shadows,%
  shapes.misc,% wg. rounded rectangle
  shapes.arrows,%
	shapes.callouts,%
  shapes%
}

\theoremstyle{plain}
\newtheorem{thm}{Theorem}[section]
\newtheorem{lem}[thm]{Lemma}
\newtheorem{prop}[thm]{Proposition}
\newtheorem{cor}[thm]{Corollary}

\theoremstyle{definition}
\newtheorem{defn}[thm]{Definition}
\newtheorem{conj}[thm]{Conjecture}
\newtheorem{exmp}[thm]{Example}
\newtheorem{assum}[thm]{Assumptions}

%\theoremstyle{remark}
\newtheorem{rem}{Remark}
\newtheorem{note}{Note}

\makeatletter
\def\th@plain{%
  \thm@notefont{}% same as heading font
  \itshape % body font
}
\def\th@definition{%
  \thm@notefont{}% same as heading font
  \normalfont % body font
}
\makeatother
\date{}
\title{Lecture 17: Connectedness}
\author{}

\begin{document}
\maketitle

Now that we have properly defined open, closed sets and limit points in a topological space, we can proceed to define the properties of 
\textit{connectedness} and \textit{compactness} for arbitrary topological space. The properties of connectedness and compactness are useful in proving 
three basic theorems about continuous functions which lie at the core of calculus. These are the following:

\begin{enumerate} [i)]
 \item \textit{Intermediate Value Theorem}.  If $f:[a, b] \to \R$ is a continuous function and if $r$ is a real number between $f(a)$ and $f(b)$, then 
 there exists an element $c \in [a, b]$ such that $f(c) = r$.
 
 \item \textit{Maximum Value Theorem}. If $f:[a,b] \to \R$ is continuous, then there exists an element $c \in [a, b]$ such that $f(x) \le f(c)$
 for every $x \in [a, b]$.
 
 \item \textit{Uniform Continuity Theorem}. If $f:[a,b] \to \R$ is continuous, then given $epsilon > 0$, there exists $\delta > 0$ 
 such that $|f(x_{1}) - f_{x_{2}}| \le \epsilon$ for every pair of numbers $x_{1}, x_{2}$ of $[a, b]$ for which $|x_{1} - x_{2}| < \delta$.
\end{enumerate}


\section{Connected Spaces}

Coarsely speaking, a topological space is said to be connected if it is not separable which implies that the space cannot be broken up into two disjoint open sets.
A more formal definition follows.

\begin{defn} 
Let $X$ be a topological space. A \textbf{separation} of $X$ is a pair $U, V$ of disjoint nonempty open sets of $X$ whose union is $X$.
The space $X$ is \textbf{connected} if there does not exist a separation of $X$. 
\end{defn}

Connectedness is a topological property, since it is formulated entirely in terms of the collection of open sets in $X$. 

\begin{rem}
If the topological space $X$ is connected, then so is any space homeomorphic to $X$. [Show proof] 
\end{rem}

An alternate definition of connectedness is the following: \\
\textit{A space $X$ is connected if and only if the only subsets of $X$ that are both open and closed in $X$ are empty set and $X$ itself.}
\begin{proof}
 Let $X$ be not connected but separable. Then, there exists open sets $U$ and $V$ which form a separation of $X$. The, by definition of separability, 
 $U$ is an open set which is neither empty nor equal to $X$. Also since $U = X \backslash V$, U is closed as well. Likewise, $V$ is also both open 
 and closed. Conversely, let us assume that there exists a set $A$ which is neither empty nor equal to $X$ which is both open and closed in $X$. 
 Then set $X \backslash A$ is an open subset of $X$. Also, $A$ and $X \backslash A$, together form a separation for $X$ and hence set $X$ is not 
 connected. Thus, we have proved the contrapositive of the reverse statement.
\end{proof}

For a subspace $Y$ of $X$, we now define of connectedness as follows.

\begin{thm}
 If $Y$ is a subspace of $X$, a separation of $Y$ is a pair of disjoint nonempty sets $A$ and $B$ whose union is $Y$, neither of which 
 contain a limit point of the other. The space $Y$ is connected if there exists no separation of $Y$. 
\end{thm}
\begin{proof}
 Let $A$ and $B$ form the separation of $Y$. We need to show that $A$ and $B$ do not contain each other's limit points. We first show that 
 $B$ does not contain any limit points of $A$. Since $\bar{A} \cap Y$ is the closure of $A$ in $Y$, we need to show that its intersection 
 with $B$ is an empty set. Here, $\bar{A}$ is the closure of $A$ in $X$. Since $A$ is also closed in $Y$, we have $A = \bar{A} \cap Y$.
 But since $A$ and $B$ are disjoint by hypothesis, $\bar{A} \cap Y$ is also disjoint with $B$. Hence, we are done. Similarly, we can show that 
 $A$ does not contain any limit points of $B$.
 
Conversely, suppose that $A$ and $B$ are disjoint nonempty sets whose union is $Y$, neither of which contains a limit point of the other,
which means $\bar{A} \cap B = \phi$. Along with the facts that $ A \cap B = \phi$ and $(A \cap B) \subset (\bar{A} \cap B)$, we conclude that 
$A = \bar{A}$ i.e, $A$ is closed. Likewise, we can show that $B$ is also closed. Since $B = Y \backslash A$ and $A = Y \backslash B$, both 
$A$ and $B$ are open in $Y$ as well. 
\end{proof}

\begin{exmp}
Let $X$ denote a two-point space in in-discrete topology $\mathcal{T}\{\phi, X\}$. Clearly, there is no separation of $X$, so $X$ is connected. 
\end{exmp}

\begin{exmp}
Let $Y$ denote the subspace $[-1 0) \cup (0, 1]$ on the real line $\R$. Each of the sets $[-1, 0)$ and $(0, 1]$
are nonempty and open in $Y$ (although not in $\R$); therefore, they form a separation of $Y$. Alternatively, note that 
neither of these sets contains a limit point of the other.
\end{exmp}

\begin{exmp}
 Let $X$ be the subspace $[-1, 1]$ of the real line. The sets $[-1, 0]$ and $(0, 1]$
 are disjoint and nonempty, but they do not form a separation of $X$, because the first set 
 is not open in $X$. Alternatively, first set contains the limit point $0$, of the second set.
 Indeed, there exists no separation of the space $[-1, 1]$.
\end{exmp}

\begin{exmp}
 The set of rational $\mathbb{Q}$ on the real line is also not connected. The only connected subspaces of 
 $\mathbb{Q}$ are the one point sets: If $Y$ is a subspace of $\mathbb{Q}$, containing two points $p$ and $q$,
 one can choose an irrational number $a$ lying between $p$ and $q$, and write $Y$ as a union of open sets 
 \begin{equation}
  Y \cap (- \infty, a) \; \text{ and } \; Y \cap (a, \infty).    
  \nonumber
 \end{equation}
\end{exmp}

\begin{thm} \label{thm_connected_set_enclosure}
 If the sets $C$ and $D$ form a separation of $X$, and if $Y$ is a connected subspace of $X$, then $Y$ lies entirely within 
 either $C$ or $D$.
\end{thm}
\begin{proof}
 Since $C$ and $D$ form a separation of $X$, the sets $C$ and $D$ are open in $X$ and consequently the sets $C \cap Y$ and $D \cap Y$
 are open in $Y$. Since $C$ and $D$ are disjoint, so are $C \cap Y$ and $D \cap Y$. Also, $(C \cap Y) \cup (D \cap Y) = (C \cup D) \cap Y = X \cap Y = Y$,
 the pair $\{(C \cap Y), (D \cap Y) \}$ constitute a valid separation for $Y$. From connectedness of $Y$, one of them must be empty and therefore, the other one 
 must contain the entire set $Y$. Hence, $Y$ has to lie completely inside either $C$ or $D$.
\end{proof}

Now, we will try to build more sophisticated connected sets by combining smaller connected sets. Following theorem shows us how to do so.

\begin{thm} \label{thm_union_preserves_connectedness}
 The union of collection of connected subspaces of $X$ that have a point in common is connected.
\end{thm}
\begin{proof}
Let $\{A_{\alpha}\}$ be a collection of connected subspaces of space $X$. Let $p$ be a point which is common to 
all $A_{\alpha}$ i.e., $p \in A_{\alpha}$. We need to show that $Y = \cup_{\alpha}A_{\alpha}$ is a connected space.
We prove this via contradiction. Let us assume that $C$ and $D$ form the separation of subspace $Y$ in $X$. Since 
$A_{\alpha} \subseteq Y$, and $A_{\alpha}$ is connected, by Theorem \ref{thm_connected_set_enclosure}, $A_{\alpha}$ must lie completely
inside either $C$ or $D$. Same goes for all $\alpha$. Let there be $\alpha_{1}$ and $\alpha_{2}$ such that $A_{\alpha_{1}}$ and $A_{\alpha_{2}}$
belong to $C$ and $D$ respectively. Since $C \cap D = \phi$, $A_{\alpha_{1}}$ and $A_{\alpha_{2}}$ cannot have a common element, which is 
a contradiction. Therefore, said $\alpha_{1}$ and $\alpha_{2}$ do not exist and there is no non-trivial separation of $Y$, and hence it is connected.  
\end{proof}


\begin{thm}
Let $A$ be a connected subspace of $X$. If $A \subset B \subset \bar{A}$, then $B$ is also connected.
\end{thm}
\begin{proof}
Suppose that $C$ and $D$ constitute the separation of $B$. Since $A$ is a connected subset of $B$, $A$ must lie completely inside either $C$
or $D$. Without loss of generality, let $A \subset C$. This implies that $\bar{A} \subset \bar{C}$. Since $D$ does not contain any limit points of 
$C$, we have $\bar{C} \cap D = \phi$. Consequently, $\bar{A} \cap D = \phi$. Since $B \subset \bar{A}$, $B$ and $D$ do not intersect, implying 
that $D$ is empty. This contradicts our assumption that $C$ and $D$ form a separation for $B$. Thus, $B$ has to be connected. 
\end{proof}


\begin{thm}
The image of a connected space under a continuous map is connected.
\end{thm}
\begin{proof}
Let $f: X \to Y$ be a continuous map; let $X$ be connected. Then, we need to show that the image space $Z = f(X)$ is also connected.
Let $g: X \to Z$ be a map obtained from $f$ by restricting its range to $Z$, Then, $g$ is also a continuous and surjective map. Let $Z = A \cup B$ 
be a separation of $Z$ into disjoint nonempty open sets in $Z$. Then, $g^{-1}(A)$ and $g^{-1}(B)$ are also disjoint open sets in $X$, 
since $g$ is a continuous map. Also $g^{-1}(A)$ and $g^{-1}(B)$ are nonempty as $g$ is surjective. Therefore, $g^{-1}(A)$ and $g^{-1}(B)$
form a nontrivial separation of $X$, contradicting the assumption that $X$ is connected.
\end{proof}

\begin{thm}
A finite cartesian product of connected spaces is connected.
\end{thm}
\begin{proof}
We start by proving the theorem for two connected spaces $X$ and $Y$. Choose a ``base point'' $a \times b$ in the product space $X \times Y$. 
Note that the horizontal slice $X \times b$ is connected as a consequence of it being homeomorphic with $X$ via an extension map $f: X \to 
X \times \R$, which maps $x \in X$ to $(x, b)$. Likewise, each vertical slice $a \times Y$ is homeomorphic with $Y$. As a result, the 
``T-shaped'' space $T_{x} (X \times b) \cup (a \times Y)$ is also connected (Theorem \ref{thm_union_preserves_connectedness}). Now, proceed
to form the union $\cup_{x} T_{x}$, which is also connected as it is a union of connected spaces with a common point $a \times b$. Since 
this union equals $X \times Y$, the space $X \times Y$ is also connected.

The proof for any finite number of connected spaces follows similarly by induction, using that fact that $X_{1} \times X_{2} \dots X_{n}$
is homeomorphic with $(X_{1} \times X_{2} \dots X_{n-1}) \times X_{n}$
\end{proof}

It is natural to ask whether the above theorem extends to arbitrary products of connected spaces. The answer depends on which topology is 
used for the product, as the following example shows.
\begin{exmp}
Consider the Cartesian product $\R^{\infty}$ in the box topology. We can write $\R^{\infty}$ as the union of the 
set $A$ consisting of all bounded sequences of real numbers, and the set $B$ of all the unbounded sequences. 
These sets are disjoint, and each is open in the box topology. 

If $a$ is a point of $\R^{\infty}$, the open set
\begin{equation*}
 U = (a_{1}-1, a_{1} + 1) \times (a_{1}-1, a_{1} + 1) \times \dots
\end{equation*}
consists entirely of bounded sequences if $a$ is bounded, and of unbounded sequences if $a$ is unbounded. Thus, even through $\R$ is connected, 
$\R^{\infty}$ is clearly not connected as we have demonstrated its nontrivial separation.
\end{exmp}

\end{document}                                                                                                                                                                                                  