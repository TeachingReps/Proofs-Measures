\documentclass[a4paper,english,12pt]{article}
\usepackage{%
	amsmath,%
	amsfonts,%
	amssymb,%
	amsthm,%
	hyperref,%
	url,%
	latexsym,%
	epsfig,%
	graphicx,%
	psfrag,%
	subfigure,%	
	color,%
	tikz,%
	pgf,%
	pgfplots,%
	pgfplotstable,%
	pgfpages,%
	proofs%
}

\usepgflibrary{shapes}
\usetikzlibrary{%
  arrows,%
	backgrounds,%
	chains,%
	decorations.pathmorphing,% /pgf/decoration/random steps | erste Graphik
	decorations.text,%
	matrix,%
  positioning,% wg. " of "
  fit,%
	patterns,%
  petri,%
	plotmarks,%
  scopes,%
	shadows,%
  shapes.misc,% wg. rounded rectangle
  shapes.arrows,%
	shapes.callouts,%
  shapes%
}

\theoremstyle{plain}
\newtheorem{thm}{Theorem}[section]
\newtheorem{lem}[thm]{Lemma}
\newtheorem{prop}[thm]{Proposition}
\newtheorem{cor}[thm]{Corollary}

\theoremstyle{definition}
\newtheorem{defn}[thm]{Definition}
\newtheorem{conj}[thm]{Conjecture}
\newtheorem{exmp}[thm]{Example}
\newtheorem{assum}[thm]{Assumptions}

%\theoremstyle{remark}
\newtheorem{rem}{Remark}
\newtheorem{note}{Note}

\makeatletter
\def\th@plain{%
  \thm@notefont{}% same as heading font
  \itshape % body font
}
\def\th@definition{%
  \thm@notefont{}% same as heading font
  \normalfont % body font
}
\makeatother
\date{}

%opening
\title{Lecture 16: The subspace topology, Closed sets}
\author{Parimal Parag}

\begin{document}
\maketitle

\section{Closed Sets and Limit Points}
\begin{defn}
A subset $A$ of a topological space $X$ is said to be \textbf{closed} if the set $X - A$ is open.
\end{defn}
\begin{thm}
Let $Y$ be a subspace of $X$ . Then a set $A$ is closed in $Y$ if and only if
it equals the intersection of a closed set of $X$ with $Y$
\begin{proof}
Assume that $A = C \cap Y$ , where $C$ is closed in $X$ . Then $X-C$ is open in $X$ , so that $(X - C) \cap Y$ is open in $Y$ , by definition of the subspace topology. But $(X - C) \cap Y = Y - A$. Hence $Y - A$ is open in $Y$ , so that $A$ is closed in $Y$ . Conversely, assume that $A$ is closed in $Y$ .Then $Y - A$ is open in $Y$ , so that by definition it equals the intersection of an open set $U$ of $X$ with $Y$ . The set $X - U$ is closed in $X$ , and $A = Y \cap (X - U )$, so that $A$ equals the intersection of a closed set of $X$ with $Y$ , as desired. .
\end{proof}
\end{thm} 
\begin{thm}
Let $Y$ be a subspace of $X$ . If $A$ is closed in $Y$ and $Y$ is closed in $X$ ,
then $A$ is closed in $X$ .
\end{thm}
\begin{defn}
Given a subset $A$ of a topological space $X$ , the \textbf{interior} of $A$ is defined as the union of all open sets contained in $A$, and the \textbf{closure} of $A$ is defined as the intersection of all closed sets containing $A$.
\[ Int A =\cup\{U\subseteq A | U\in \mathcal{T}  \}\]
\[Closure \  \bar{A}= \cap \{\mathcal{F} \subseteq A | X - \mathcal{F} \in \mathcal{T}\}
\]
If $A$ is open, $A = Int A$; while if $A$ is closed, $A = \bar{A}$; furthermore
\[ IntA \subseteq A \subseteq \bar{A}
\]
\end{defn}
\begin{thm}
Let $Y$ be a subspace of $X$ ; let $A$ be a subset of $Y$ ; let $\bar{A}$ denote the
closure of $A$ in $X$ . Then the closure of $A$ in $Y$ equals $\bar{A} \cap Y$.
\begin{proof}
Let $B$ denote the closure of $A$ in $Y$ . The set $\bar{A}$ is closed in $X$ , so $\bar{A} \cap Y$ is closed in $Y$ by Theorem 1.2. Since $A \cap Y$ contains $A$, and since by definition $B$ equals the intersection of all closed subsets of $Y$ containing $A$, we must have $B \subset \bar{A} \cap Y$. On the other hand, we know that $B$ is closed in $Y$. Hence by Theorem 1.2, $B = C \cap Y$ for some set $C$ closed in $X$ . Then $C$ is a closed set of $X$ containing $A$; because $\bar{A}$ is the intersection of all such closed sets, we conclude that $\bar{A} \subset C$. Then $(\bar{A} \cap Y ) \subset (C \cap Y ) = B$.
\end{proof}
\end{thm}
\begin{defn}
A \textbf{neighborhood} of a point $x \in X$ is an open set $U$ containing $x$
\end{defn}
\begin{thm}
Let $A$ be a subset of the topological space $X$ .
\begin{enumerate}
\item  Then $x \in \bar{A}$ if and only if every open set $U$ containing $x$ intersects $A$ .
\item Supposing the topology of $X$ is given by a basis, then $x \in \bar{A}$ if and only if every basis element $B$ containing $x$ intersects $A$ .
\end{enumerate}
\begin{proof}
Consider the statement in $1$. It is a statement of the form $P \leftrightarrow Q$. Let
us transform each implication to its contrapositive, thereby obtaining the logically
equivalent statement $(not P) \leftrightarrow (not Q)$. Written out, it is the following:
\[x \notin \bar{A} \longleftrightarrow \ there \ exists \ an \ open \ set \ U \ containing \ x \ that \ does \ not \ intersect \ A \]
the set $U = X - \bar{A}$ is an In this form, our theorem is easy to prove. If $x$ is not in $A$, open set containing $x$ that does not intersect $A$, as desired. Conversely, if there exists an open set $U$ containing $x$ which does not intersect $A$, then $X - U$ is a closed set containing $A$. By definition of the closure $A$, $x$ cannot be in $A$. Statement $2$ follows readily. If every open set containing $x$ intersects $A$, so does every basis element $B$ containing $x$, because $B$ is an open set. Conversely, if every basis element containing $x$ intersects $A$, so does every open set $U$ containing $x$, because $U$ contains a basis element that contains $x$.
\end{proof}
\end{thm}
\begin{exmp}
\begin{enumerate}
\item Let $X$ be the real line $\mathbb{R}$. If $A = (0, 1]$, then $\bar{A} = [0, 1]$, for every neighborhood of $0$ intersects $A$, while every point outside $[0, 1]$ has a neighborhood disjoint from A
\item If $B = \{1 \ n | n \in  \mathbb{Z}_+ \}$, then $\bar{B}= \{0\} \cup B$.
\item if $C = \{0\} \cup  (1, 2), \ then \  \bar{C}= \{0\} \cup [1, 2]$.
\item If $\mathbb{Z}_+ $ is the set of positive integers, then $\mathbb{\bar{Z}}_+ = \mathbb{Z}_+$
\end{enumerate}
\end{exmp}
\begin{defn}
If $A$ is a subset of the topological space $X$ and if $x$ is a point of $X$ , we say that $x$ is a \textbf{limit point} (or “cluster point,” or “point of accumulation”) of $A$ if every neighborhood of $x$ intersects $A$ in some point other than $x$ itself.
$x$ is a limit point of $A$ if it belongs to the closure of $A - \{x\}$
\end{defn}
\begin{exmp}
\begin{enumerate}
\item Consider the real line $\mathbb{R}$. If $A = (0, 1]$, then the point $0$ is a limit point of $A$ and so is the point $0.5$ . In fact, every point of the interval $[0, 1]$ is a limit point of $A$, but no other point of $\mathbb{R}$ is a limit point of $A$.
\item If $B = \{1/n | n \in \mathbb{Z}_+\}$, then $0$ is the only limit point of $B$. Every other point $x$ of $\mathbb{R}$ has a neighborhood that either does not intersect $B$ at all, or it intersects $B$ only in the point $x$ itself. If $C = \{0\} \cup (1, 2)$, then the limit points of $C$ are the points of the interval $[1, 2]$.
\item If $\mathbb{R}_+$ is the set of positive reals, then every point of $\{0\} \cup \mathbb{R}_+$ is a limit point of $\mathbb{R}_+$.
\end{enumerate}
\end{exmp}
\begin{thm}
Let $A$ be a subset of the topological space $X$ ; let $A'$ be the set of all
limit points of $A$ . Then
\[\bar{A} = A \cup A'
\]
\begin{proof}
If $x$ is in $A'$ , every neighborhood of $x$ intersects $A$ (in a point different from $x$). Therefore, by Theorem 1.7, $x$ belongs to $\bar{A}$. $A' \subset \bar{A}$. Since by definition $A \subset \bar{A}$ it follows that $A \cup A' \subset \bar{A}$.
To demonstrate the reverse inclusion, we let $x$ be a point of $\bar{A}$ and show that
$x \in  A \cup A'$ . If $x$ happens to lie in $A$, it is trivial that $x \in A \cup A'$ ; suppose that $x$ does not lie in $A$. since $x \in  \bar{A}$. we know that every neighborhood $U$ of $x$ intersects $A$; because $x \notin A$, the set $U$ must intersect $A$ in a point different from $x$. Then $x \in  A'$ ,so that $x \in A \cup A'$ , as desired.
\end{proof}

\end{thm}
\section{Housdorff Spaces}
\begin{defn}
Let $(X,\mathcal{T})$ be topological space. A sequence $\{x_n : n \in \mathbb{N}\} \subseteq X$ \textbf{converges} to $x_0 \in X$ if for all $U$ neighborhood of $x_0 \  \exists \  \mathbb{N}$ such that $x_n \in U$ for all $n \geq \mathbb{N}$
\end{defn}
\begin{defn}
A topological space $X$ is called a\textbf{ Hausdorff space} if for each pair $x_1$ , $x_2$
of distinct points of $X$ , there exist neighborhoods $U_1$ and $U_2$ of $x_1$ and $x_2$ , respectively that are disjoint.
\end{defn}
\begin{thm}
Every finite point set in a Hausdorff space $X$ is closed.
\begin{proof}
It suffices to show that every one-point set $\{x_0\}$ is closed. If $x$ is a point of $X$ different from $x_0$ , then $x$ and $x_0$ have disjoint neighborhoods $U$ and $V$ , respectively. Since $U$ does not intersect $\{x_0\}$, the point $x$ cannot belong to the closure of the set $\{x_0\}$. As a result, the closure of the set $\{x_0\}$ is $\{x_0\}$ itself, so that it is closed.
\end{proof}
\end{thm}
\begin{exmp}
the real line $\mathbb{R}$ in the finite complement topology is not a
Hausdorff space, but it is a space in which finite point sets are closed. The condition that finite point sets be closed has been given a name of its own: it is called the \textbf{$T_1$ axiom}.
\end{exmp}
\begin{thm}
Let $X$ be a space satisfying the $T_1$ axiom; let $A$ be a subset of $X$. Then the point $x$ is a limit point of $A$ if and only if every neighborhood of $x$ contains infinitely many points of $A$ .
\begin{proof}
 If every neighborhood of $x$ intersects $A$ in infinitely many points, it certainly intersects $A$ in some point other than $x$ itself, so that $x$ is a limit point of $A$. Conversely, suppose that $x$ is limit point of $A$, and suppose some neighborhood $U$ of $x$ intersects $A$ in only finitely many points. Then $U$ also intersects $A - \{x\}$ in finitely many points; let $\{x_1 , . . . , x_m \}$ be the points of $U \cap (A - \{x\})$. The set $X - \{x_1 , . . . , x_m\}$ is an open set of $X$ , since the finite point set $\{x_1 , . . . , x_m \}$ is closed;
then
\[U \cap (X - \{x_1 , . . . , x_m \})
\]
is a neighborhood of $x$ that intersects the set $A - \{x\}$ not at all. This contradicts the assumption that $x$ is a limit point of $A$.
\end{proof}
\end{thm}
\begin{thm}
If $X$ is a Hausdorff space, then a sequence of points of $X$ converges
to at most one point of $X$ .
\begin{proof}
Suppose that $x_n$ is a sequence of points of $X$ that converges to $x$. If $y = x$, let $U$ and $V$ be disjoint neighborhoods of $x$ and $y$, respectively. Since $U$ contains $x_n$ for all but finitely many values of $n$, the set $V$ cannot. Therefore, $x_n$ cannot converge
to $y$.
\end{proof}
\end{thm}
\begin{defn}
If the sequence $x_n$ of points of the Hausdorff space $X$ converges to the point $x$ of $X$ , we often write $x_n \rightarrow x$, and we say that $x$ is the \textbf{limit} of the sequence $x_n$.
\end{defn}


	
\end{document}