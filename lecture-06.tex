\documentclass[a4paper,english,12pt]{article}   	% use "amsart" instead of "article" for AMSLaTeX format
\usepackage{%
	amsmath,%
	amsfonts,%
	amssymb,%
	amsthm,%
	hyperref,%
	url,%
	latexsym,%
	epsfig,%
	graphicx,%
	psfrag,%
	subfigure,%	
	color,%
	tikz,%
	pgf,%
	pgfplots,%
	pgfplotstable,%
	pgfpages,%
	proofs%
}

\usepgflibrary{shapes}
\usetikzlibrary{%
  arrows,%
	backgrounds,%
	chains,%
	decorations.pathmorphing,% /pgf/decoration/random steps | erste Graphik
	decorations.text,%
	matrix,%
  positioning,% wg. " of "
  fit,%
	patterns,%
  petri,%
	plotmarks,%
  scopes,%
	shadows,%
  shapes.misc,% wg. rounded rectangle
  shapes.arrows,%
	shapes.callouts,%
  shapes%
}

\theoremstyle{plain}
\newtheorem{thm}{Theorem}[section]
\newtheorem{lem}[thm]{Lemma}
\newtheorem{prop}[thm]{Proposition}
\newtheorem{cor}[thm]{Corollary}

\theoremstyle{definition}
\newtheorem{defn}[thm]{Definition}
\newtheorem{conj}[thm]{Conjecture}
\newtheorem{exmp}[thm]{Example}
\newtheorem{assum}[thm]{Assumptions}

%\theoremstyle{remark}
\newtheorem{rem}{Remark}
\newtheorem{note}{Note}

\makeatletter
\def\th@plain{%
  \thm@notefont{}% same as heading font
  \itshape % body font
}
\def\th@definition{%
  \thm@notefont{}% same as heading font
  \normalfont % body font
}
\makeatother
\date{}
\geometry{letterpaper}                   		% ... or a4paper or a5paper or ... 
%\usepackage[parfill]{parskip}    		% Activate to begin paragraphs with an empty line rather than an indent


\title{Lecture 6: Functions : Injectivity, Surjectivity, and Bijectivity}
\author{}

\begin{document}
\maketitle

\section {Injectivity, Surjectivity, Bijectivity}
We are interested in finding out the conditions for a function to have a left inverse, or right inverse, or both.
\begin{defn} Let $ f \colon A \to B $ be a function.
\begin{enumerate} 
	\item The map $f$ is \textbf{injective} (also called one-to-one/monic/into) if $x \neq y $ implies $f(x) \neq f(y) $ for all $x,y \in A$. Equivalently, $f(x) = f(y) $ implies $x = y $ for all $x,y \in A$.
	\item The map $f$ is \textbf{surjective} (onto/epic) if for every $b \in B$ , there exists some $a \in A $ such that $f(a)=b$, equivalently $f(A)=B$.
 \item The map $f$ is \textbf{bijective} if it is both injective and surjective.
\end{enumerate}
\end{defn}
\begin{lem} Let $f: A \to B$ be a function. Then the following are true.
\begin{enumerate}[i)]
	\item Function $f$ is injective iff $ f^{-1}(\{b\})$ has at most one element for all $b \in B$ .
	\item Function $f$ is surjective iff $ f^{-1}(\{b\})$ has at least one element for all $b \in B$ .
	\item Function $f$ is bijective iff $ f^{-1}(\{b\})$ has exactly one element for all $b \in B$ .
\end{enumerate}
\end{lem}
%\subsection{special case}
\begin{exmp}
A function $f \colon \mathbb{R} \to \mathbb{R}$ on real line is a special function.  This function is injective iff any horizontal line intersects at at most one point, surjective iff any horizontal line intersects at at least one point, and bijective iff any horizontal line intersects at exactly one point.
\end{exmp}

In the following lemma, we see that injectivity, surjectivity, and bijectivity is preserved by composition of functions.
\begin{lem}
Let $f \colon A \to B$ , $g \colon B \to C$ be functions.
\begin{enumerate}[i)]
\item  Functions $f, g$ are injective, then function $f \circ g$ injective.
\item  Functions $f, g$ are surjective, then  function $f \circ g$ surjective.
\item  Functions $f, g$ are bijective, then function $f \circ g$ bijective.
\end{enumerate}
\end{lem}

In the following theorem, we show how these properties of a function are related to existence of inverses.
\begin{thm}\label{Thm:BijectiveInverse}
Let $A$ and $B$ be two non-empty sets and let $f \colon A \to B$ be a function. 
\begin{enumerate}[i)]
\item Function $f$ has a right inverse iff $f$ is surjective.
\item Function $f$ has a left inverse iff $f$ is injective.
\item Function $f$ has a inverse iff $f$ is bijective.
\end{enumerate}
\end{thm}
\begin{proof} Let $A$ and $B$ be non-empty sets and $f: A \to B$ a function.
\begin{enumerate}[i)] 
	\item $\Rightarrow$. Suppose $f$ has a right inverse $g$, then $f \circ g$ = $1_{B}$. We will show $f$ is surjective. Let $b \in B$, we need to find an element $a \in A$ such that $f(a)=b$. Let $a=g(b)$ then $f(a)  = (f \circ g)(b) =1_{B}(b)=b$.\\
$\Leftarrow$. Suppose $f$ is surjective. We wish to show that $f$ has a right inverse, i.e., there exists a map $g : B \to A$ such that $f \circ g$=$1_{B}$. We define $g$ as follows. For each $b \in B$ , let $g(b)=a$ for some $a \in f^{-1}(\{b\})$. Notice that $f^{-1}(\{b\})$ is non-empty for all $b \in B$ since $f$ is surjective. Now $(f \circ g)(b)=b$ for all $b \in B$ by definition. Hence $f \circ g = 1_{B}$.
	\item $\Rightarrow$. Suppose $f$ has a left inverse $h$, then $h \circ f$ = $1_{A}$. %We will show $f$ is injective. 
	Let $x,y \in A$ such that $f(x) = f(y) \in B$. By definition of left inverse we have then $x = (h \circ f)(x) = (h \circ f)(y) = y$. Hence, $f$ is injective.\\
$\Leftarrow$. Suppose $f$ is injective. We wish to show that $f$ has a left inverse, i.e., there exists a map $h \colon B \to A$ such that $h \circ f$=$1_{A}$. We define $h: B \to A$ as follows. For each $b \in f(A)$, let $h(b)=f^{-1}(\{b\})$. Notice that $f^{-1}(\{b\})$ has exactly one element for all $b \in f(A)$ since $f$ is injective. Now $(h \circ f)(a)=a$ for all $a \in A$ by definition. Hence $h \circ f = 1_{A}$.
	\item Follows from first two parts.
\end{enumerate}
\end{proof} 


\begin{thm} Let A and B be non empty sets and let $f \colon A \to B$ be a function.
\begin{enumerate}[(i)]
\item the function $f$ is surjective iff $g \circ f = h \circ f$ implies $g=h$ for all functions $g,h \colon B \to X$ for all sets X.
\item The function $f$ is injective iff $f \circ g = f \circ h$ implies $g=h$ for all functions $g,h \colon Y \to A$ for all sets Y.
\end{enumerate}
\begin{proof} Let A and B be non empty sets and let $f \colon A \to B$ be a function.
\begin{enumerate}
\item $\Rightarrow$. Assume $f$ surjective. Let $g,h \colon B \to X$ such that $g \circ f = h \circ f$ for some set X. By Theorem~\ref{Thm:BijectiveInverse}(i), function $f$ has right inverse $q \colon B \to A$. By associativity, we have $(g \circ f) \circ q = g \circ (f \circ q)$ and thus $g \circ 1_{B} = h \circ 1_{B}$ and so $g=h$.\\
$\Leftarrow$. We assume $f$ not surjective. Let $b \in B$ such that $f^{-1}(\{b\})= \emptyset$. Let $X=\{1,2\}$ define $g,h \colon B \to X$ by $g(y)=1$ for all $y \in B$ and $h(y)=1$ for all $y \in B \setminus \{b\}$ and $h(b)=2$. It is clear that $g \circ f = h \circ f$, even though $g \neq h$.
\item $\Rightarrow$. Assume $f$ injective. Let $g,h \colon Y \to A$ such that $f \circ g = f \circ h$ for some set Y. By Theorem~\ref{Thm:BijectiveInverse}(ii), function $f$ has left inverse $q \colon B \to A$. By associativity, we have $ q \circ (f \circ g) = (q \circ f) \circ g$. Therefore, thus $1_{A} \circ g = 1_{A} \circ h$ and so $g=h$.\\
$\Leftarrow$. We assume $f$ not injective. Let $x \neq y \in A$ such that $f(x) = f(y)$. Let $Y = \{1,2\}$ define $g,h \colon Y \to A$ by $g(1) = x, g(2) = y$. It is to see that $f \circ g = f \circ h$, even though $g \neq h$.
\end{enumerate}
\end{proof}
\end{thm}
\section{Sets of functions}
\begin{defn} Let A and B be sets. The set $\mathcal{F}(A,B)$ is defined to be the set of all functions $f \colon A \to B$. That is,
\begin{align*}
\mathcal{F}(A,B) = \{f \colon A \to B |\, f \,\text{ is a function }\}.
\end{align*}
\end{defn}
\begin{rem} If $A$ and $B$ are not empty, then $\mathcal{F}(A,B)$ is non-empty.
\end{rem}
\begin{exmp} 
\begin{enumerate}
\item Let $A=\{1,2\}$ and $B=\{x,y\}$. Then, $\mathcal{F}(A,B) = \{ (f,g,h,k)\}$, where $f=\{(1,x),(2,x)\}$ , $g=\{(1,y),(2,y)\}$ , $h=\{(1,x),(2,y)\}$ , $k=\{(1,y),(2,x)\}$. These functions are illustrated in Figure~\ref{Fig:fg} and Figure~\ref{Fig:hk} respectively.
\begin{figure}[hhhh]
\centering
\scalebox{.8}{
 
 \begin{tikzpicture}[ele/.style={fill=black,circle,minimum width=.8pt,inner sep=1pt},every fit/.style={ellipse,draw,inner sep=-1pt}]
  \node[ele,label=left:$1$] (a1) at (0,4) {};    
  \node[ele,label=left:$2$] (a2) at (0,3) {};    
 

  \node[ele,,label=right:$x$] (b1) at (4,4) {};
  \node[ele,,label=right:$y$] (b2) at (4,3) {};
 

  \node[draw,fit= (a1) (a2) ,minimum width=2cm] {} ;
  \node[draw,fit= (b1) (b2) ,minimum width=2cm] {} ;  
  \draw[->,thick,shorten <=2pt,shorten >=2pt] (a1) -- (b1);
  \draw[->,thick,shorten <=2pt,shorten >=2pt] (a2) -- (b1);
 
 \end{tikzpicture} \begin{tikzpicture}[ele/.style={fill=black,circle,minimum width=.8pt,inner sep=1pt},every fit/.style={ellipse,draw,inner sep=-1pt}]
  \node[ele,label=left:$1$] (a1) at (0,4) {};    
  \node[ele,label=left:$2$] (a2) at (0,3) {};    
 

  \node[ele,,label=right:$x$] (b1) at (4,4) {};
  \node[ele,,label=right:$y$] (b2) at (4,3) {};
 

  \node[draw,fit= (a1) (a2) ,minimum width=2cm] {} ;
  \node[draw,fit= (b1) (b2) ,minimum width=2cm] {} ;  
  \draw[->,thick,shorten <=2pt,shorten >=2pt] (a1) -- (b2);
  \draw[->,thick,shorten <=2pt,shorten >=2pt] (a2) -- (b2);
 
 \end{tikzpicture}}
\caption{Functions $f = \{(1,x),(2,x)\}$, $g = \{(1,y),(2,y)\}$.}
\label{Fig:fg}
\end{figure}
\begin{figure}[hhhh]
\centering
\scalebox{.8}{\begin{tikzpicture}[ele/.style={fill=black,circle,minimum width=.8pt,inner sep=1pt},every fit/.style={ellipse,draw,inner sep=-1pt}]
  \node[ele,label=left:$1$] (a1) at (0,4) {};    
  \node[ele,label=left:$2$] (a2) at (0,3) {};    
 

  \node[ele,,label=right:$x$] (b1) at (4,4) {};
  \node[ele,,label=right:$y$] (b2) at (4,3) {};
 

  \node[draw,fit= (a1) (a2) ,minimum width=2cm] {} ;
  \node[draw,fit= (b1) (b2) ,minimum width=2cm] {} ;  
  \draw[->,thick,shorten <=2pt,shorten >=2pt] (a1) -- (b1);
  \draw[->,thick,shorten <=2pt,shorten >=2pt] (a2) -- (b2);
 
 \end{tikzpicture} \begin{tikzpicture}[ele/.style={fill=black,circle,minimum width=.8pt,inner sep=1pt},every fit/.style={ellipse,draw,inner sep=-1pt}]
  \node[ele,label=left:$1$] (a1) at (0,4) {};    
  \node[ele,label=left:$2$] (a2) at (0,3) {};    
 

  \node[ele,,label=right:$x$] (b1) at (4,4) {};
  \node[ele,,label=right:$y$] (b2) at (4,3) {};
 

  \node[draw,fit= (a1) (a2) ,minimum width=2cm] {} ;
  \node[draw,fit= (b1) (b2) ,minimum width=2cm] {} ;  
  \draw[->,thick,shorten <=2pt,shorten >=2pt] (a1) -- (b2);
  \draw[->,thick,shorten <=2pt,shorten >=2pt] (a2) -- (b1);
 
 \end{tikzpicture}}
\caption{Functions $h=\{(1,x),(2,y)\}$ , $k=\{(1,y),(2,x)$.}
\label{Fig:hk}
\end{figure}
\item Let $\mathcal{F}(\mathbb{R}, \mathbb{R})$ be set of real-valued functions on real line. Let $C(\mathbb{R}, \mathbb{R})$ and $D(\mathbb{R}, \mathbb{R})$ be set of real-valued continuous and derivable functions respectively. Then, $F(\mathbb{R},\mathbb{R}) \supset C(\mathbb{R},\mathbb{R}) \supset D(\mathbb{R},\mathbb{R})$. In fact, let $k \colon D(\mathbb{R},\mathbb{R}) \to F(\mathbb{R},\mathbb{R})$ such that $k(f)=f'$. Then, $k$ is not injective. For example, let $f(x) = x^2 + 5$ and $g(x) = x^2 + 7$ for all $x \in \mathbb{R}$. Then, $k(f) = k(g)$, but $f \neq g$.
\item Let $f \in \mathcal{F}(\mathbb{N},\mathbb{R})$, then $f(1),f(2),\ldots$ is a sequences of real numbers. That is, $\mathcal{F}(\mathbb{N},\mathbb{R})$ set of real-valued sequences.
\end{enumerate}
\end{exmp}

\begin{lem}
Let $A,B,C,D$ be sets suppose that there are bijective maps $f \colon A \to C$ and $g \colon B \to D$ then there is a bijective map between $\mathcal{F}(A,B)$ and $ \mathcal{F}(C,D)$.
\end{lem}
\begin{proof} Since $f$ and $g$ are bijective maps , They have inverse maps $f^{-1}$ and $g^{-1}$ respectively.
Define $\Phi \colon \mathcal{F}(A,B) \to \mathcal{F}(C,D)$ by $\Phi (h) = g \circ h \circ f^{-1}$. for all $h \in \mathcal{F}(A,B)$, it's easy to see that $\Phi (h) \in \mathcal{F}(C,D)$ for all $h \in \mathcal{F}(A,B)$. We need to show $\Phi$ is bijective. One can see the corresponding commutative diagram in Figure~\ref{Fig:CommDiagBij}.

\underline{\textbf{Injective:}} Let $h,k \in \mathcal{F}(A,B)$ and suppose $\Phi (h)=\Phi (k)$. Then $g \circ h \circ f^{-1} = g \circ k \circ f^{-1}$. Hence, from associativity of composition of functions we have, $h = g^{-1} \circ ( g \circ h \circ f^{-1}) \circ f = g^{-1} \circ ( g \circ k \circ f^{-1} ) \circ f = k$.

\underline{\textbf{Surjective:}} Let $r \in \mathcal{F}(C,D)$, and define $t = g^{-1} \circ r \circ f$. Clearly, $t \in \mathcal{F}(A,B)$, and $\Phi(t) = g \circ (g^{-1} \circ r \circ f) \circ f^{-1} = r$.
\end{proof}
\begin{figure}[hhhh]
\centering
\scalebox{1}{\begin{tikzpicture}
  \matrix (m) [matrix of math nodes,row sep=3em,column sep=4em,minimum width=2em]
  {
     A & B \\
     C & D \\};
  \path[-stealth]
    (m-1-1) edge node [left] {f} (m-2-1)
            edge node [below] {h} (m-1-2)
    (m-2-1) edge node [below] {$\Phi (h) $} (m-2-2)
     (m-1-2) edge node [right] {g} (m-2-2);
 \end{tikzpicture}}
\caption{Commutative diagram for bijection between functional spaces $\mathcal{F}(A,B)$ and $ \mathcal{F}(C,D)$, where $f:A \to C$ and $g: B \to D$ are bijective functions.}
\label{Fig:CommDiagBij}
\end{figure}

We can use set of function to study power sets. Indeed, we will show there is a bijection between these two in the following proposition.
\begin{prop} Let A be a non empty set. Then, there is a bijective map from $F(A,\{0,1\})$ to $\mathcal{P}(A)$.
\end{prop}
\begin{proof} Let $\Phi \colon \mathcal{P}(A) \to \mathcal{F}(A,\{0,1\})$ be defined as follows. Let $S \in \mathcal{P}(A)$, then $\Phi(S) \colon A \to \{0,1\}$ is defined as
\begin{equation*}
[\Phi (S)](x) = \begin{cases}1, &  x \in S,\\0,  & x \notin S.\end{cases}
\end{equation*}
We will show that $\Phi$ is bijection in two different ways.
\begin{enumerate}
\item We can show that $\Phi$ is injective and surjective.
\textbf{Injectivity:} Let $S,T \in \mathcal{P}(A)$ and suppose $\Phi (S)=\Phi (T)$, we will show $S=T$. Let $y \in S$, then $[\Phi (S)](y) = 1 = [\Phi (T)](y)$ then $y \in T$. Hence, we have shown $S \subseteq T$. Similarly, we can show, $T \subseteq S$.\\
\textbf{Surjectivity:} Let $f \in \mathcal{F}(A,\{0,1\})$. Define $S \in \mathcal{P}(A)$ such that $S = \{x \in A | f(x) = 1\} = f^{-1}(\{1\})$. We will show that $\Phi(S) = f$. For all $x \in S$, we have $\Phi(S)(x) = 1 = f(x)$. On the other hand, for all $x \notin S$, we have $\Phi(S)(x) = 0 = f(x)$.
\item We will construct an explicit inverse for $\Phi$. To this end, we define $\Psi : \mathcal{F}(A, \{0,1\}) \to \mathcal{P}(A)$ as
\begin{equation*}
\Psi(f) = f^{-1}(\{1\}) = \{x \in A : f(x) = 1\}, \text{ for all } f \text{ in } \mathcal{F}(A, \{0, 1\}).
\end{equation*}
We will show $\Psi \circ \Phi = 1_{\mathcal{P}(A)}$ and $\Phi \circ \Psi = 1_{\mathcal{F}(A, \{0,1\})}$. Let $S \in \mathcal{P}(A)$, then $(\Psi \circ \Phi)(S) = [\Phi(S)]^{-1}(\{1\}) = S$. Now, let $f \in \mathcal{F}(A, \{0,1\})$, such that $(\Phi \circ \Psi)(f) = \mathbbm{1}_{\{f^{-1}(\{1\})\}}$. Clearly, $f(x) = \mathbbm{1}_{\{f^{-1}(\{1\})\}}(x)$ for all $x \in A$.
\end{enumerate}
\end{proof}

\section{Product of an indexed family of sets}
We can define Cartesian product of two sets as $A \times B = \{(a,b), a \in A , b\in B\}$ in terms of $\{f\in \mathcal{F}(\{1,2\}, A \cup B)$, such that $f(1) \in A$, and $f(2)\in B\}$.
\begin{defn}
Let $I$ be a non empty set and let $\{ A_i : i \in I \}$ be a family of sets indexed by $I$. The \textbf{product} of all the sets in this family of sets is defined as 
\begin{equation*}
\prod_{i\in I} A_i = \{ f \in \mathcal{F}(I,\bigcup_{i \in I}A_i) : f(i) \in A_i, \text{ for all }i \in I\}.
\end{equation*}
If all the sets $A_i$ are equal to a single set $A$. Then, we denote $\prod_{i \in A_i}A_i$ by $A^I$.
\end{defn}
\begin{prop} If $I \neq \emptyset$ and $A$ is a set, then $A^I = f(I,A)$.
\end{prop}
\begin{proof} Follows from the definition.
\end{proof}
\end{document}  