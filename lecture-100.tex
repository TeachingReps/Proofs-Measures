\documentclass[a4paper,english,12pt]{article}
\usepackage{%
	amsmath,%
	amsfonts,%
	amssymb,%
	amsthm,%
	hyperref,%
	url,%
	latexsym,%
	epsfig,%
	graphicx,%
	psfrag,%
	subfigure,%	
	color,%
	tikz,%
	pgf,%
	pgfplots,%
	pgfplotstable,%
	pgfpages,%
	proofs%
}

\usepgflibrary{shapes}
\usetikzlibrary{%
  arrows,%
	backgrounds,%
	chains,%
	decorations.pathmorphing,% /pgf/decoration/random steps | erste Graphik
	decorations.text,%
	matrix,%
  positioning,% wg. " of "
  fit,%
	patterns,%
  petri,%
	plotmarks,%
  scopes,%
	shadows,%
  shapes.misc,% wg. rounded rectangle
  shapes.arrows,%
	shapes.callouts,%
  shapes%
}

\theoremstyle{plain}
\newtheorem{thm}{Theorem}[section]
\newtheorem{lem}[thm]{Lemma}
\newtheorem{prop}[thm]{Proposition}
\newtheorem{cor}[thm]{Corollary}

\theoremstyle{definition}
\newtheorem{defn}[thm]{Definition}
\newtheorem{conj}[thm]{Conjecture}
\newtheorem{exmp}[thm]{Example}
\newtheorem{assum}[thm]{Assumptions}

%\theoremstyle{remark}
\newtheorem{rem}{Remark}
\newtheorem{note}{Note}

\makeatletter
\def\th@plain{%
  \thm@notefont{}% same as heading font
  \itshape % body font
}
\def\th@definition{%
  \thm@notefont{}% same as heading font
  \normalfont % body font
}
\makeatother
\date{}

%opening
\title{Lecture 100: Measures}
\author{}

\begin{document}
\maketitle

\section{Measures}
\begin{defn}%[Measure]
Let $(X, \F)$ be a measurable space. A set mapping $\mu: \F \to [0, \infty]$ is called a \textbf{measure} if 
\begin{enumerate}[i.]
	\item $\mu (\emptyset) = 0$,
	\item \textbf{Countable additivity.} $\mu(\cup_{n \in \N}A_n) = \sum_{n \in \N} \mu(A_n)$ for all sequences $\{A_n: n \in \N\}$ of pairwise disjoint sets in $\F$.
\end{enumerate}
\end{defn}

%\begin{defn}%[Finite Additivity]
%Let $(X, \F)$ be a measurable space and $\{A_i : i \in [n] \}$ a finite set of pairwise disjoint sets in $\F$, then a set mapping $\mu: \F \to [0, \infty]$ is called \textbf{finite additive} if $\mu(\cup_{i = 1}^nA_i) = \sum_{i = 1}^n \mu(A_i)$.
%\end{defn}
\begin{defn}%[Measure Space]
Let $X$ be a non-empty set, equipped with a $\sigma$-algebra $\F$, and $\mu$ is a measure on $\F$, then $(X, \F, \mu)$ is called a \textbf{measure space}.
\end{defn}
\begin{thm}[Properties of Measures] Let $(X, \F, \mu)$ be a measure space, and $\{A_n: n \in \N \} \subseteq \F$, then the following are true.
\begin{enumerate}
	\item \textbf{Finite additivity.} If $\{A_i: i \in [n]\}$ a finite collection of pairwise disjoint sets, then $\mu(\cup_{i = 1}^nA_i) = \sum_{i = 1}^n\mu(A_i)$.
	\item \textbf{Monotonicity.} If $A_i \subseteq A_j$, then $\mu(A_i) = \mu(A_j)$.
	\item \textbf{Continuity from below.} If $\{A_n: n \in \N\}$ is an increasing sequence of sets, i.e. $A_n \subseteq A_{n+1}$ for all $n \in \N$, then 
	\begin{align*}
	\mu\left(\bigcup_{n \in \N} A_n\right) = \lim_{n \in \N}\mu(A_n) = \sup_{n \in \N}\mu(A_n).
	\end{align*}
	\item \textbf{Continuity from below.} If $\{A_n: n \in \N\}$ is a decreasing sequence of sets, i.e. $A_{n+1} \subseteq A_{n}$ for all $n \in \N$ and $\mu(A_1) < \infty$, then
	\begin{align*}
	\mu\left(\bigcap_{n \in \N} A_n\right) = \lim_{n \in \N}\mu(A_n) = \inf_{n \in \N}\mu(A_n).
	\end{align*}
	\item \textbf{Countable sub-additivity.} $\mu\left(\bigcup_{n \in \N}A_n\right) = \sum_{n \in \N} \mu(A_n)$.
\end{enumerate}
\end{thm}
\begin{proof} Let $(X, \F, \mu)$ be a measure space, and $\{A_n: n \in \N \} \subseteq \F$.
\begin{enumerate}
	\item Let $\{B_i: i \in \N\}$ such that $B_i = A_i$ for $i \in [n]$ and $B_i = \emptyset$ for $i \in \N \setminus [n]$. Then, $\{B_i: i \in \N\}$ is a countable collection of pair-wise disjoint sets, where $\cup_{i \in \N}B_i = \cup_{i=1}^n A_i$. Hence, it follows from the definition of measure, that 
	\begin{align*}
	\mu(\bigcup_{i=1}^n A_i) = \mu(\bigcup_{i \in \N}B_i) = \sum_{i \in \N}\mu(B_i) = \sum_{i=1}^n\mu(A_i).
	\end{align*}
	\item We can write $A_j = A_i \cup (A_j \setminus A_i)$, a finite union of disjoint sets. Therefore from finite additivity and non-negativity of measure, we have
	\begin{align*}
	\mu(A_j) = \mu(A_i) +  \mu(A_j \setminus A_i) \geq \mu(A_i).
	\end{align*}
	\item If $\{A_n: n \in \N\}$ is an increasing sequence of sets, then $\lim_{n \in \N}A_n = \cup_{n \in \N}A_n$. We construct a sequence of pair-wise disjoint sets $\{B_n: n \in \N\}$ inductively such that $B_1 = A_1$ and $B_n = A_n \setminus A_{n-1}$ for $n > 1$. It follows that for all $n \in \N$,
	\begin{align*}
	\bigcup_{i \leq n}B_i = \bigcup_{i \leq n}A_i = A_n. 
	\end{align*}
	From $\sigma$-additivity of measures, it follows that
	\begin{align*}
	\mu\left(\bigcup_{n \in \N} A_n\right) &= \mu(\bigcup_{n \in \N} B_n) = \sum_{n \in \N}\mu(B_n) = \lim_{n \in \N}\sum_{i=1}^n\mu(B_i)\\&= \lim_{n \in \N}\mu(\bigcup_{i=1}^nB_i) = \lim_{n \in \N}\mu(A_n).
	\end{align*}
	\item If $\{A_n: n \in \N\}$ is a decreasing sequence of sets, then we can form increasing sequence of sets $\{B_n: n \in \N\}$ such that $B_n = A_1 \setminus A_n$. From DeMorgan's law, we have	
	\begin{align*}
	\left(\bigcup_{n \in \N} B_n\right) = \left(\bigcup_{n \in \N} A_1 \setminus A_n\right) = \left(A_1 \setminus \left(\bigcap_{n \in \N} A_n\right)\right) = A_1 \setminus \left(\bigcap_{n \in \N} A_n\right).
	\end{align*} 
	Applying finite-additivity of measures to sets $A_1 \setminus \left(\bigcap_{n \in \N} A_n\right)$ and $\left(\bigcap_{n \in \N} A_n\right)$, we get
	\begin{align*}
	\mu\left(\bigcup_{n \in \N} B_n\right) = \mu(A_1) - \mu\left(\bigcap_{n \in \N} A_n\right).
	\end{align*} 
	Now applying continuity from below to this increasing sequence of sets, we get 
	\begin{align*}
	\mu(A_1) - \mu\left(\bigcap_{n \in \N} A_n\right) = \mu\left(\bigcap_{n \in \N} B_n\right) = \lim_{n \in \N}\mu(B_n) = \lim_{n \in \N}\mu(A_1) - \mu(A_n).
	\end{align*}
	Since $\mu(A_1)$ is finite, we can subtract it from both sides of the above equation to get the result.
	\item We construct a pair-wise disjoint sequence of sets $\{B_n: n \in \N\}$ inductively from sets $\{A_n: n \in \N\}$, such that $B_n = A_n \setminus \cup_{i=1}^{n-1}A_i$. It follows from the construction that $\cup_{i=1}^nB_i = \cup_{i=1}^nA_i$, and from monotonicity that $\mu(B_n) \leq \mu(A_n)$ for all $n \in \N$. Hence, we conclude that
	\begin{align*}
	\mu\left(\bigcup_{n \in \N} A_n\right) = \mu\left(\bigcup_{n \in \N} B_n\right) = \sum_{n \in \N}\mu(B_n) \leq \sum_{n \in \N}\mu(A_n).
	\end{align*}
\end{enumerate}
\end{proof}	

\begin{rem} 
We can replace finiteness of $\mu(A_1)$ by finiteness of $\mu(A_n)$ for some $n \in \N$. We can just take $B_j = A_n \setminus A_j$ for all $j \in \N$.
\end{rem}
\begin{rem} Finite assumption is necessary in continuity from above. It is possible that $\mu(A_j) = \infty$ for all $j \in \N$ and $\mu\left(\bigcap_{n \in \N}A_n\right)  < \infty$. %One such example is when $A_n = \{m \in \N: m \geq n\}$.
\end{rem}
\section{Limits of Sets}
\begin{defn} Let $\{A_n: n \in \N\}$ be a countable collection of subsets of a non-empty set $X$. We can define increasing sequence of sets $\{B_n: n \in \N\}$ and decreasing sequence of sets $\{C_n: n\in \N\}$, such that 
\begin{align*}
B_n = \bigcup_{k \geq n}A_k \text {  and } C_n = \bigcap_{k \geq n}A_k.
\end{align*}
Then, we can define \textbf{limit superior} and \textbf{limit inferior} of sets $\{A_n: n \in \N\}$ as 
\begin{xalignat*}{3}
&\lim_{n \in \N}\sup A_n = \bigcap_{n \in \N}B_n,&&\lim_{n \in \N}\inf A_n = \bigcup_{n \in \N}C_n.
\end{xalignat*}
When $\lim\sup A_n = \lim\inf A_n$, we say that \text{limit} $\lim A_n$ of the sequences of set exists and 
\begin{align*}
\lim A_n = \lim\sup A_n = \lim\inf A_n.
\end{align*}
\end{defn}
\begin{rem} From the definition it is clear that the following hold.
\begin{align*}
\lim_{n \in \N}\sup A_n &= \{x \in X: x \in A_n \text{ for infinitely many } n\}.\\
\lim_{n \in \N}\inf A_n &= \{x \in X: x \in A_n \text{ for all but finitely many } n\}.
\end{align*}
\end{rem}
\begin{prop}\label{Prop:IncreasingSet} Let $\{A_n: n \in \N\}$ be a countable collection of increasing subsets of a non-empty set $X$. That is, $A_n \subseteq A_{n+1}$ for all $n \in \N$. Then,
 \begin{align*}
\lim\inf A_n = \lim\sup A_n = \bigcup_{n \in \N} A_n.
\end{align*}
\end{prop}
\begin{proof} Let $B_n = \cup_{k \geq n}A_k = \cup_{k \in \N}A_k$ independent of $n$. Further, $C_n = \cap_{k \geq n}A_k = A_n$. Hence, we have
\begin{align*}
\lim\inf A_n = \bigcup_{n \in \N}A_n = \bigcap_{n \in \N}B_n = \lim\sup A_n.
\end{align*}
\end{proof}
\begin{prop}\label{Prop:DecreasingSet} Let $\{A_n: n \in \N\}$ be a countable collection of decreasing subsets of a non-empty set $X$. That is, $A_{n+1} \subseteq A_{n}$ for all $n \in \N$. Then,
\begin{align*}
\lim\inf A_n = \lim\sup A_n = \bigcap_{n \in \N} A_n.
\end{align*}
\end{prop}
\begin{proof} Let $B_n = \cup_{k \geq n}A_k$, then $B_n = A_n$ since $\{A_n: n \in \N\}$ is a decreasing sequence of sets.  On the other hand, $C_n = \cap_{k \geq n}A_k = \cap_{k \in \N}A_k$, that is independent of $n$. Hence, we have 
\begin{align*}
\lim\inf A_n = \bigcup_{n \in \N}C_n = \bigcap_{k \in \N}A_k = \lim\sup A_n.
\end{align*}
\end{proof}
\begin{prop} Let $\{A_n: n \in \N\}$ be a countable collection of subsets of a non-empty set $X$. Then,
\begin{align*}
\lim\inf A_n \subseteq \lim\sup A_n.
\end{align*}
\end{prop}
\begin{proof} Let $B_n$ and $C_n$ be as in definition. Then, it is easy to see that $\{B_n: n \in \N\}$ and $\{C_n: n \in \N\}$ are decreasing and increasing sequences of sets respectively. Further, we have $C_n \subseteq A_m \subseteq B_m$ for all $m \geq n$. Further, since $B_m$ is decreasing sequence of sets, we have for all $n \in \N$,
\begin{align*}
C_n \subseteq \cap_{m \geq n} B_m = \cap_{m \in \N} B_m = \lim\sup A_n.
\end{align*}
\end{proof}
\end{document}